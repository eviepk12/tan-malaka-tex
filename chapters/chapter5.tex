\section{Destitution}

How many thousands, or even how many hundreds of thousands of Indonesians are 
sleeping on empty stomachs in the public halls every day as they unwind is not clear. 
The government has a complete record of the figures of profitable farms and enterprises, 
especially the names of those who are liable to pay taxes, but it forgets to give 
any certainty about the livelihood of the people as a whole. It is true that sometimes a 
committee is set up by the government, but this body does not represent the people, and of 
course it never indicts big capital, though it criticises it. "Orderly" and "independent" 
examinations, as evidence of good intentions, have never been heard of.\nline

If we want to know how many industrial, plantation and transport workers 
there are, we will know how many "colonial slaves" are starving in Indonesia, 
because most of the industrial workers are poor, because they have to sell or 
lease their land to the big companies, thus losing their land and livelihood.\nline

It cannot be blamed on the negligence and indifference of the government. 
Although we are working with inadequate figures, this does not mean that the 
plight of the Indonesian people is a closed book to us; on the contrary, 
it cannot even be surmised that two to three million oppressed slaves receive 
wages just enough to survive starvation. The largest part of them are organised. 
They are, for example, railway labourers, sweepers, porters and brakemen, who start 
work with a salary of f 15 - with one to two rupiahs increase each year - and 
reach a maximum of f 30 to f 40 a month when they are already greying of old age. 
That's too little a salary in the age of capitalism, and it's especially sad, considering 
that the care and responsibility of a group of labourers are dependent on the lives of thousands of people.\nline

If hundreds of thousands of unorganised sugar workers do not have the courage 
to ask for an increase in their wages; if the landless peasantry work only a 
few months a year for 30 or 40 cents a day, that is, when cutting sugar cane; 
if 250 to 300 thousand indentured workers - the so-called "free slaves" in East 
Sumatra - earn 30 to 40 cents a day, who dares to say that in these days a man 
(even if he is an inlander!), with his daughter, can live as a human being on 
12 to 25 rupiah a month? If anyone says so, he is a donkey or, more insultingly, a "traitor".\nline

The ironworkers, a group of labourers whose wages are high in other countries, 
are underpaid in Surabaya, living conditions of a dog cage, with insufficient food, 
clothing and other necessities of life, until they fall prey to Chinese and Arab loan sharks. 
We still hear that their salaries were between 30 and 40 rupiah. In Surabaya, which is known as 
a city of commerce, this salary means only a barrier against certain death.\nline

What is the name of the governor-general who one day shamefully told us that there were 
thousands of coolies in the port of Jakarta, because their wages were not enough to rent 
the huts so beloved by the Javanese? So shameful and uncertain is the fate of the labourers 
who are still working, what about the unemployed who are becoming more and more numerous?\nline

In the \emph{Verslag van de Suiker Enquete Commissie} (p. 99) we read the very significant sentence: 
"Presumably half of the families of the people on the island of Java belong to landowners, 
and the rest live from the enterprise and trade of bumiputra or not. There must be 
hundreds of thousands of people who have nothing, who occasionally work for one of 
the farmers and who misleadingly call themselves farmers". Moreover, in the cities 
there are many people who loiter along the streets, eating a bite in the morning and 
a bite in the evening. We do not have any comprehensive, correct and valid statistics on how many there are.\nline

But anyone who has lived in one of the many sugar towns such as Banyumas, Solo, Kediri 
and Surabaya, and who has really paid attention to the lives of the people, will be 
astonished at the "patience" and "endurance" of the people in bearing their hardships, 
knowing that taxes are far beyond the means of the people, a known fact to government officials.\nline

All and every living person (even if he does not earn a living) must pay taxes. 
We could include quotations from all the relevant parties, but since we consider them to be useless, we don't need to add them here.\nline

(In passing we might say that the big industries and trading partnerships also pay taxes. 
However, this is a matter of mere convention, because in various ways the tax can be brought 
down on the heads of the destitute and disenfranchised Indonesians).\nline

Padoux, the Chinese government's advisor in the \emph{"Memorandum for the National Commission 
for the Study of Financial Problems"}, determined that individual persons in the Philippines, 
Indo-China, France, Siam, Indonesia, and China paid taxes of \$7.50, 8.50, 9.50, 15.50, and 1.20 respectively.\nline

Thus, the highest tax is in Indonesia! It is twice that of the Philippines, 
almost twice that of Indo-China, France, and twelve times that of China. 
This calculation is based on comparisons made before 1923. Back then there was still 
\emph{"Inlandsch Verponding"} - a shameless travesty - the likes of which no despotic king in Java had ever done.\nline

\% research Inlandsch Verponding

Mr Yeekes explains in "de Opbouw" (1923) that the average income 
of the Indonesian people is f 196 a year. Of this income much has 
to be spent as a taxpayer, and outside Java for labour as well, 
so that a month's income is only f 13. A figure far below the minimum. 
Mr Yeekes' calculation is for the whole of Indonesia, so the income of 
the people in Central Java must be even less.\nline

We in modern times are saddened and shocked to see Javanese people 
living in shabby huts or no shelter at all, starving and dressed in 
filthy rags, living in the most dangerous climate of Indonesia, lacking 
health care, due to epidemics of malaria, tapeworm, cholera and pestilence; 
"only" hundreds of thousands died when these diseases were rampant.\nline

A fortitude worthy of praise!

\section{Darkness}

Still, the Dutch "government of peasants and shopkeepers" is as afraid of the 
Universities and Colleges as it is of ghosts. Still he has not escaped the menace 
of "intellectual labour". They have mistaken the political attitude of British doctrine 
and drawn the wrong conclusions. They are too ignorant to think that it was 
because of the insight and skill of British imperialism that there was an 
intelligentsia in India which in times of trouble often helped the British 
government, and that it was because of the intellectual class, including extremists, 
that Tilak and Mahatma Gandhi won economic victories with their extensive boycott movements. 
And it was also because the British co-operated with the modern indigenous bourgeoisie, on 
the political and economic fronts, that the British were able to continue to rule in India 
despite the recent onslaught of the noncooperation movement.\nline

The Dutch government in the debates always raised various objections to the establishment 
of universities in Indonesia, objections that could only be accepted by children. All of 
these arguments were only used at the time of colonisation and can be summarised in the following reasons.\nline

\begin{enumerate}
    \item That this government, having repented, should now make itself the educator of the Indonesian people at the people's own expense and should give the best possible instruction to the children of Indonesia, if it is not fond of nonsense.
    \item That the Indonesian people are neither superior in brain nor in spirit, nor inferior to any other nation, and that they are ripe for any kind of teaching.
    \item That the first Indonesian university need not be a copy of a European university, but a European university based on the spiritual intelligence and circumstances of Indonesian society at this time.
\end{enumerate}

The Philippines - with 12 million people - already has four universities and several colleges, 
but Indonesia with five times as many people does not have a single one.\nline

We do not forget for a moment that if the "Dutch" establish a university in 
Indonesia, its teaching will undoubtedly be higher than in other colonies, 
just as they say that Dutch universities are much higher than universities 
anywhere else. Regardless of this selfish behaviour, we just want to say to the Dutch, " Show your competence in Indonesia first!."\nline

"It's the act that you have to actually prove!"

But apart from money, which for a Dutchman weighs more than political aspirations and reasons, 
there are other political views that we would not expect from the ignorant Dutchman.\nline

Not long ago Mr. Hardeman, the head of the Department of Education, explained in a session 
of the People's Council that establishing a university did not necessarily produce educated 
workers, because the need for educated workers was temporarily decreasing, due to economic 
difficulties which would later of course be restored. With this the "specter" of the Java Bode of 30 June has disappeared.\nline

We will come back to the consequences of Dutch educational policy later on. Here we want to 
show, with figures, that the lower, middle and higher education institutions have always 
been insufficient for a population of 55 million. This must be recognised regardless of 
the empty excuses of the so-called "government".\nline

Let's skip over the higher institutions that have been churning out dozens of doctors, masters, 
and engineers for years. Let us turn for a moment to the question of the lower schools. 
The number of children who had to attend school in 1919 was as follows: H.I.S. 1\%, 
Public Schools 5\%, Village Schools 8\% to 14\%. Approximately 86\% of the children 
who should have gone to school had no place \emph{(according to the 1923 report of the 
N.I.O.G. congress published in the Indische Courant)}. Those who can read and write 
are now estimated at 5\% to 6\%, perhaps even 2\% to 3\%.\nline

The total expenditure of the colleges in 1919 was reportedly f 20,000,000 and f 75,000,000 
for 150,000 European children and f 12,500,000 for the children of 55,000,000 Indonesian 
taxpayers. In 1923 the expenditure of the college was f 34,452,000. Thus, 30 cents was 
spent on a native child at that time, which was 1/7th of what was spent on a Filipino child.\nline

For other institutions which set a good example to the dissatisfied people, such as the police, 
the military and the fleet, an expenditure of f 156,274,000 was made in that year. In addition, 
f 300,000,000 will be spent in another year. This was a very heavy burden on the shoulders of the miserable people.\nline

We, as revolutionaries, were determined in 1921 to correct the government's negligence 
in education by establishing our own schools. By going through all kinds of difficulties, 
such as technical, personnel, financial, political and police difficulties, we were finally 
able to establish throughout Java 52 schools with about 50,000 pupils and the number is growing. 
However, the schools were violently suppressed. Teachers were banned from teaching all the 
time for petty reasons, and parents were frightened. The final blow was dealt by the Green 
Union (a band of thieves mobilised, hired and led by the government and its minions). 
These hired thugs were told to burn down schools, frighten and persecute people, students 
and teachers. And the orders were carried out by them in earnest.\nline

A vibrant people's movement towards the eradication of illiteracy led joyfully and effortlessly 
by revolutionaries in Priangan in 1922 met with a fate that was just as terrible.\nline

The politics of this government in matters of education can be summarised with the words: 
"the Indonesian nation, must remain ignorant so that public peace and security are maintained".\nline

\section{Savagery and Slavery}

Despite 300 years of exposure to Western civilisation, our people still live 
in a state of ignorance and lack of rights. The peasants have never had a day 
of certainty about their ownership, independence or even their lives. Every year 
the people's tax cycle turn harder and harder. The workers are not allowed to 
hold associations or express their objections. The reasonable requests of the 
people are not heard. The education and leaders of the people who are trusted 
by the people are labelled and treated as seditionists and bandits, and are 
therefore, with no prior investigation, put in jail, kept in rat rooms, 
banished out of the country or beaten to death. Reasoned requests and protests 
are crushed by a bureaucracy that apparently prefers to drown in its own rot.\nline

Now let us allow the famous Prof Van Vollenhoyen to speak and 
denounce the attitude of the Dutch government, as written in his 
book \emph{Indonesier en zijn Grond}. Indonesia may have at least 70\% of 
the population living on the agricultural industry; and for this reason it is important 
for a well educated man - whose honour and position have never been compromised - 
to hear what has been done to the peasantry in recent years by a power which claims 
to be the "guardians of the people" and feels that it is doing so.\nline

We do not want to dig up what has already happened, so let's first talk about the events of 
the past 60 years of the last century. Everybody knows and justifies the saying that in 
those years "the Javanese were persecuted". However, not everyone can readily see what kind 
and to what extent the peasant property was exploited. To find this out, we do not need to 
read the books of the Dutch government's cruelty as "instigators and spreaders of hatred", but simply take the matter into our own hands.\nline

Daendels' despotism, rotten as it was, could still be considered extraordinary. 
He had his own authority over the people's rice fields and fields to pay bumiputra employees.\nline

Van Vollenhoven goes on to say: 
"whereas the Javanese kingship, which is almost as rotten as ours, is 'confined' to its own kingdoms, Kedu, 
Yogyakarta and Surakarta, we extend it to cover the whole island".\nline

The village officials take what is good for themselves and give what is bad to the foolish people. All of this is arbitrary.\nline

"What do we expect now?" van Vollenhoven asked next. Are we gradually going to stop 
fussing about the rice fields because of the land tax (this is already happening). 
Are we gradually going to stop taking away the rice fields and plantations under 
coercion of the people (this has already happened). Will we reduce and eliminate 
the harmful effects of forced labour on lands belonging to the people (this is already happening). 
And then we will learn to keep our itching hands quiet. The latter has not yet happened.\nline

If in 1919 a Javanese whose land rights had been damaged by f 1,000 came to complain to 
the controller, he would be sentenced to eight days' hard labour. If he went to the President of 
the District Court, he would be told, "There is no time!" and if he went to the \emph{Wali Negeri}\footnote[6]{Wali Negeri is a general term for local landlords/aristocracy} 
for protection, "His Majesty is not pleased to answer". In rather polite Dutch it was called "godsgeklaagd".

It often happens that in the midst of a piece of land that the government intends 
to give to the big landlords there is an indigenous owned land. According to the law, 
the land cannot be taken unless it is for the government's own use. In practice, 
however, people would try to persuade the inlander to exchange his rights to the land for money.
The following is the conclusion of Prof van. Vollenhoven's conclusions, which cannot be faulted for their truth and reality.\nline

"But it seems that this is what is most important to the Indonesian who owns their own land, 
it is very difficult to have any feeling other than that of the constant abuse; the lying and 
deception of their legal land rights on paper, as an inexhaustible effort to deprive them of 
their rights or to prevent them from being able to utilise them".\nline

And aren't all these tax increases now a cruel abuse of power if 
we are to use Prof van Vallenhoven's own words? Are our people 
informed when the government takes a decision and discusses our property, jobs and freedom?\nline

Never! Just as the government never asks us, "Do we like it or not?"\nline

The 55 million Indonesians do not have a single representative in this government 
who can voice their opinions or advice, protest or denounce. The militarist and 
bureaucratic thugs who suck our blood and control our destiny, we never liked or 
voted for. We cannot stop them because we have no political power. We must fight 
them if we don't like them, nothing else! In conclusion, all the rules and regulations 
that rule us in Indonesia are made at their own will and the payment of taxes is, in theory or practice, "theft".\nline

Let us consider the plight of these 300,000 indentured workers, 
who are "supposedly" protected by the government. A wage of approximately f12 a 
month is barely enough to buy clothes which are usually shredded, because they 
are worn every day working in the fields. A day's work is 14 to 18 hours, because 
the tobacco fields are usually far from the workers lodgings, more accurately the 
workers cages, even though the contract only says 10 hours.\nline

The treatment of European plantation supervisors is better described as stabbings, mutilations; 
assaults and murders of assistants and "subtlety disturbed to the point of cruelty!" 
Here was a social scene poisoned by gambling, opium and prostitution that degraded 
the character of the workers and left them so much in debt to their employers that their contracts had to be renewed forever.\nline

Such working conditions are - directly or indirectly - imposed on the mostly 
illiterate and ignorant peasantry; they are pressured into a "contract" recognised 
by the government. The contract states that they are "not allowed to organise and 
go on strike" - by which means they can demand wages and working conditions that are 
slightly better than in other countries. This is recognised by the government. 
It is something that only the "slave traders" of the barbaric age can defend.\nline

Let us remember the crimes committed in Deli. Let us remember the recent atrocities 
committed by Europeans in Lampung and South Sumatra, crimes that were regarded as fairy 
tales in the Middle Ages. Even more than a fairy tale is the leniency of the punishment 
given by the government to these European "bastards".\nline

The hundreds of thousands or millions industrial, plantation and 
transport workers in Java and elsewhere, who are enslaved not by contract, 
who are said to be "free workers", fare no better than the original indentured 
slaves. One by one, their feet are bound with chains of rules, so that they 
cannot organise and fight against the exploitative capitalists. In the People's 
Council, the Upper and Lower Houses, and the newspapers of different stripes, 
the right of organisation and the right to strike of the Indonesian workers have 
been repeatedly discussed! There is no need to repeat them here, or to elaborate 
on these forced laws. Once again, it is said that these laws are not according to 
modern principles, but are coercive laws imposed by a group of bureaucrats on 
the Indonesian workers, in order to bind all their efforts towards the improvement of their conditions.\nline

All these laws that are being enforced cause us to remember the dark ages of 
savagery and slavery. So many laws are forced upon the political movement that 
we cannot openly say or write anything about the colonisers or open the eyes of the enslaved people.\nline

The Indonesian people must keep their mouths shut in the wake of the persecution 
of leaders they trust and love, as well as when leaders are deliberately deprived 
of several months of their independence or, without proper investigation, thrown 
away because they are considered dangerous or treacherously stabbed, mutilated, 
beaten to death, or deprived of their very lives by bullets.\nline

When people are told that a beloved leader, such as Haji Misbach, was said to 
have died "of black fever" in a government-appointed captivity, whether they want to or not, they must believe it.\nline

When the people hear that an educated and decent young man, such as our Soegono, the leader 
of the V.S.T.G. who is said to have "committed suicide" in prison, while his head and hands 
bear the marks of torture and a finger is completely crushed, the people "cannot indict", nor can they protest at all.\nline

And the government, which is "said" to be the guardian and protector of 
our people, did not conduct a careful examination of the causes of the 
sudden death of a capable leader of the people who fought with an unwavering 
heart and was, in short, loved and trusted by the people. It does not care or 
have the moral courage to admit and correct its mistakes and punish the guilty according to the \emph{Fiat justitiaruate cellum law}.\nline

(Let justice be done though the heavens fall!.)

Justice in Indonesia is only for a small group of white colonisers. For the Indonesians who were entitled to the land, there was no justice and no courts.\nline