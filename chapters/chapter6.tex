The three hundred years of Dutch perfidy in the imperialistic world of 
the so-called colonisers created one-of-a-kind social and national 
contradictions throughout Asia. On the one hand, there was capital 
accumulating in highly modern agriculture, with very high production, 
and with international connections, organized into a number of syndicates 
and trusts, which made enormous profits. On the other hand, the peasantry, 
the small merchants, were turned into workers. They are huddled together as 
industrial workers in the cities and agricultural workers in the fields. All this gave birth to misery, slavery and insecurity.\nline

If the class struggle resembles an impassable valley, which in 
the Western countries and Japan has given rise to socialism, anarchism 
and bolshevism, in Indonesia the valley has been deepened by the conflict 
between the Dutch and the Indonesians. This conflict, though not an essential 
cause, could very well provoke a war of independence. The "conflict" between 
the capitalist Netherlands and the Indonesian workers is our social relations, 
which differs from that of other countries. This conflict is born in its sharpest 
form. This sharpness is due not only to the absence of modern capital from the 
Indonesian nation, but also to the differences in religion, nation, language, and culture between the coloniser and the colonised.\nline

In advanced capitalist countries, social conflict is divided into two classes: 
the class of capitalists and their followers and the class of workers. The capitalists 
are those who own land, factories, railways, ships and banks, and increase their wealth 
with the unpaid labour of the workers, which Marx describes as \emph{"met de zijn kapitaal geaccumuleenk meerwaarde" (with its capital accumulated surplus value)}. 
The workers are those whose property and land have been seized by the capitalists. They were once peasants and small traders, 
but now all their possessions are completely gone except their labour, their bodies and their lives. The price of this labour 
is "subject" to the fluctuating prices in the labour market. The capitalist lives from extortion and the workers from their 
labour. This wage is due to the "iron law" of the labour market - it cannot pay for the labour that is put in 
(because of the fierce competition in the labour market and the fear of starving to death, the labourer is forced to 
accept the lowest possible wage).\nline

The capitalist class, which is relatively smaller in number compared to the working class, 
uses "invisible weapons", such as schools, churches or mosques, and newspapers, as well 
as class tools such as the police, army, prisons, and justices. Parliament, mosques, 
churches, schools and newspapers have the power to sedate and weaken the hearts of the 
workers with poisonous education. When they are unable to do so, prisons, the police and the military are used.\nline

Economic competition among capitalists led to the formation of companies. 
They can fight their isolated enemies. If a company in "desperate" competition cannot conquer its opponent, 
it tries to organise a compromise. The two companies, who were once enemies, now become one syndicate. 
Thus they can raise the prices of their goods at will, to the detriment of the buyers (poor workers and peasants).\nline

So, a syndicate is a combination of several companies. However, the partners worked 
in their own way and were independent as usual. In order to increase their strength and 
centralise them under one leader for the economic struggle, a trust was formed. So, a 
syndicate has many chairmen, while a trust has only one, and the way it works, a trust 
can more thoroughly control the world market than a syndicate.\nline

In the markets of Western nations, especially America, we see a number of charcoal mines, 
iron industries, oil refineries and shipping companies that were once fragmented now united 
in huge trusts, headed by trust magnates. We hear names like Morgan the King of Banks, 
Rockefeiler the King of Oil, Carnagie the King of Steel and Ford the Lord of Automobiles.\nline

In Germany we saw how the many trusts were tied together into one "joint trust". 
The iron, charcoal and paper factories, the ship and railway companies were all 
subordinated to the recently deceased Stinnes. In this way, Stinnes was able to control 
the prices of raw materials and manufactured goods, as well as the transport and advertising 
costs of these manufactured goods. The formation of trusts like this was imitated by 
banks that merged themselves from companies to syndicates, from syndicates to trusts and from trusts to combined trusts.\nline

In order to continue earning interest, it appoints and dismisses heads of industry, statesmen and 
politicians, and directly or indirectly pays or bribes them. With the trust, the leadership of 
the bank company is placed in the hands of several bankers. Thus, it is the bankers who are essentially 
at the head of industry, transport, agriculture commerce, the state and politics, in short, of modern capitalist society.\nline

In light of the previously mentioned, it appears to us that the more capitalism advances, the fewer 
the wealthy and the greater the number of poor workers. In the advanced capitalistic countries such 
as Britain, Germany and America, the number of educated and skilled workers is not less than 
75\% of the population. The number of artisans but capitalised and productive is getting smaller 
and smaller. Their power and wealth are growing. The number of workers, who own nothing, is growing, 
and so is their organisation. The conflict between the stakeholders and the poor workers intensifies 
over time and eventually leads to social revolution.\nline

In Indonesia, the process of capitalisation is hardly different from the outline described above. 
Indonesian merchants and small enterprises have long since disappeared from society.\nline

Several million people now live in a state of hunger. They are landless and resourceless, 
with no hope for the future. Control over factory land, transport equipment and trade agencies 
is now all concentrated in the hands of a few syndicates such as Avros, Suikersyndikaat, 
Handeslvereeniging Amsterdam and others. The leadership of these large syndicates is in the hands of a few capitalists.\nline

The social conflict between capitalists and workers in Indonesia is - for one reason or another - 
sharper than meets the eye. The huge profits from sugar, oil, rubber, coffee, tea etc. flow mostly 
to Europe, into the pockets of the Dutch, and a small part also returns to Indonesia, but not as an 
increase in the wages of the workers, but as an addition to the existing "capital", to make it a new 
"means of exploitation". Most of the profits stayed in the Netherlands as salaries or pensions for Dutch employees.\nline

The plight of Indonesian workers can only be improved by raising their wages in proportion to the price of daily 
necessities. With the opening of several large plantations, some workers or unemployed people have found work, 
but on the other hand, their land has been leased and sold, and many farmers have lost their property. In addition, 
due to the expansion of capitalisation, everyday goods have risen in price. It is undeniable that the rise in the 
price of goods in the past ten years has not kept pace with the rise in workers wages.\nline

Thus the Indonesian people are getting poorer and poorer because their salaries remain as they have been 
(in fact, they are often lowered), while goods are getting more and more expensive. And with competition 
becoming more and more intense, because the number of people is increasing rapidly and strongly, there is 
less certainty of finding work.\nline

If the capitalists were Indonesians, the poverty and squalor would not have been so grievous because the 
remaining huge profits might have been passed on to the people. Workers' salaries might have been raised; 
teaching, people's cooperatives, industrialisation and healthcare might have been attended to and improved. 
Now none of this is happening, as the vast profits continue to be transported out of Indonesia.\nline

Apart from this draining process, social conflicts are sharpened by national differences and all 
that goes with them. The capitalists speak a different language from the people and the government is 
not a government of the people. The capitalists and the government profess other religions, have other 
morals and customs, and have different ideologies from the people. In the daily interactions between 
capitalists and workers, between the government and the people, these are very important. The Dutch 
capitalist does not know his workers; the Dutch government knows its people. It is not that he does not want to know the people.\nline

Even if he had wanted to do so, it would not have been easy for the Dutch to delve into the hearts of 
these equatorial peoples because they had not prepared the necessary factors, such as education, the 
language of social relations and the trust of the people. For this reason, the Dutch, who are said to 
be 'polite', would often use foul language against the Indonesians. The Indonesians would not like the 
Dutch government. Just as the Philippines, which was not directly subjected to the rule of the American 
Governor-General and arguably did not suffer any hardship from American officials, is still demanding 
its independence, so too the Indonesians of the South will continue to demand absolute and broad independence. 
Just as a human being does not like to be bullied and controlled by others, so do the people. They will 
eventually not allow themselves to be colonised or ruled by other nations.\nline

It is up to us to see whether the opposition between capitalist Netherlands and workers' Indonesia will remain forever or only temporarily!\nline

This opposition will gradually diminish if the present government, not the future one, 
makes major changes, economic, political and social improvements that improve the situation of all Indonesians.\nline

This right can only be achieved by establishing new industries (cotton, rubber, machine factories, shipping, mining, etc.), 
opening up large farms and multiplying roads, establishing people's cooperatives with low interest rates, giving assistance 
and materials to the peasantry, land to former poor farmers, increasing the wages of workers and reducing working hours, 
reducing or abolishing taxes and increasing taxes on plantations or large estates, and industry into a communal property, 
i.e. the government, granting the widest possible electoral rights to the indigenous people, establishing a "true" people's 
representative from which a body fully responsible to the Indonesian people will be elected, abolishing all useless bureaucratic 
bodies, such as the \emph{Raad van Indie}, \emph{de Alt gemeene Secretaris} and others.

Of course it will never happen!

Not even half of it will happen. Even if Dutch imperialism were to suddenly abandon its "petty politics" and adopt real colonial politics, 
it would be too late, too late! Dutch imperialism has neither the aspirations, the courage nor the means to bring about the
slightest meaningful change. It is too "daif" (weak) to do so and no modern native bourgeoisie can help it.\nline

As for the "foreign capital" which is dotted with a few dollars from Wall Street, 
it is merely a few pebbles thrown to fill the deep gulf between Dutch imperialism and the Indonesian people.\nline

Radical reforms like those in the Philippines can and will be carried out by America if it 
receives political power in Indonesia from the Netherlands. If this were the case, America would 
undoubtedly arrive in Indonesia in a short time with a few thousand million rupiahs. But it is 
impossible! Because it is against the interests and "honour" of the Netherlands. Because a large 
American capital in Indonesia would push Dutch capital to the side! And if the finances are tied up, 
Dutch capital is meaningless (and the Dutch are forced to become puppets of Uncle Sam).\nline

Of course the \emph{"Meneerge"} won't! Last but not least, it would mean an increase in American economic and political power 
in a strategically important part of the Pacific. This would be vigorously opposed by the spiteful British and Japanese, 
and might lead to a long and devastating world war.\nline

Therefore, for the little Dutch who are reluctant to perish, it is better to do as they please while they await their downfall. 
After all, the other colonisers (Britain, America and Japan) were better off letting the Netherlands to struggle 
with its disobedient colonies.\nline

