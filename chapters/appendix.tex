\section{Politics}

\begin{enumerate}
    \item Indonesia's immediate and absolute independence.
    \item Establish a federation of republics of the various islands of Indonesia.
    \item To immediately organize a national assembly, representing all sectors of people and religion throughout Indonesia.
    \item To immeditaely grant full voting rights to the Indonesian population, both men and women.
\end{enumerate}

\section{Economy}

\begin{enumerate}
    \item Nationalize factories, and mines, such as coal, oil, and gold mines.
    \item Nationalize forestries and large modern plantations such as sugar, rubber, tea, coffee, quinine, coconut, indigo, and casava plantations.
    \item Nationalize transport and traffic equipment.
    \item Nationalize large banks, corporations, and trading companies.
    \item Elecrification of the whole of Indonesia and establishing new industries with the help of the state, such as weaving mills, machinery, and shipping.
    \item Establish people's cooperatives by providing cheap loans by the state.
    \item Provide livestock and tools to the peasantry to improve their agriculture and establish public collective farms.
    \item Moving large numbers of people at state expense from Java to the outer islands.
    \item Distribute vacant land to the landless and poor peasants by providing financial assistance to cultivate the land.
    \item Abolish feudal remnants and landed estates and distribute the land to poor unlanded peasants.
\end{enumerate}

\section{Social}

\begin{enumerate}
    \item Set a minimum wage, seven hours of work and improve the working and living conditions of the workers.
    \item Protect workers by recognising the workers' right to strike.
    \item Workers get a share of the profits of big industries.
    \item Set up workers councils in large industries.
    \item Separate the state from the Church or Mosque and recognise religious freedom.
    \item Provide social, economic and political rights to every Indonesian citizen, both men and women.
    \item Nationalize large residences, build new residences and distribute residences to state workers.
    \item Large amount of effort shall be used to fight against infectious diseases.
\end{enumerate}

\section{Education}

\begin{enumerate}
    \item Education is compulsory and provided free of charge to all Indonesian children up to the age of 17, with Indonesian as the main   national language and English as the primary foreign language.
    \item Abolish the current system of education and establish a new system, based directly on the needs of existing or prospective industries.
    \item Improve and expand schools for artisanship, agriculture and commerce, and improve and expand technical colleges and schools for administrators.
\end{enumerate}

\section{Military}

\begin{enumerate}
    \item Abolish the imperialistic military and establish a people's militia to defend the Republic of Indonesia.
    \item Abolish the rule of living in a fortress or camp and all rules that degrade rank-and-file soldiers, and allow them to live in the villages and in the houses that will be built for them, give them good treatment and increase their salary.
    \item Granting rank-and-file soldiers the full right to organise and hold meetings.
\end{enumerate}

\section{Police and Justiciary}

\begin{enumerate}
    \item Separate the civil service, police and justiciary.
    \item Give every person indicted the full right to defend themselves before a court of law against attacks and discharge the indicted person within 24 hours, if the evidence and witnesses are insufficient.
    \item Every case having a valid ground shall be heard within five days in an open, orderly and proper court.
\end{enumerate}

\section{Action Programmes}

\begin{enumerate}
    \item Demand a seven-hour workday, minimum wage and better working conditions for the workers and their livelihoods.
    \item Recognise Trade Unions and the right to strike.
    \item Organising workers for economic and political rights.
    \item Abolish \emph{poenale sanctie}\footnote[10]{The \emph{poenale sanctie} (penal sanction) was a legal penalty in both Suriname and the Dutch East Indies. The poenale sanctie was a part of the \emph{Koelie Ordonnantie} ('Coolie Ordinance') of 1880 and stipulated that a plantation-owner could punish his coolies in any manner he saw fit, including fines. This made the plantation-owner both policeman and judge.}
    \item Abolish laws and regulations that oppress political movements, such as strike bans, press bans, meeting bans and teaching bans, and also recognise full freedom of movement.
    \item Demand the right to demonstrate, reinforced by mass demonstrations throughout Indonesia against economic and political oppression, such as against tax legislation, and demanding the immediate release of political exiles; such mass action should be reinforced by general strikes and mass disobedience.
    \item Demand the abolition of the \emph{Volksraad Raad van Indie}\footnote[11]{An advisory body for the colonial government made up mostly of dutch and indies nationals, not to be confused with the \emph{volksraad} which was a later institution borned out of the ethical policy that allowed educated aristocrats of the Indonesian population to participate.} and the \emph{Algemeene Secretaris}, and establish a National Assembly. A National Assembly from which a Workers' Council will be elected, responsible to the National Meeting.
\end{enumerate}