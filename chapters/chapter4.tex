Capitalism in Indonesia is a transplant from Europe which in 
some ways is not the same as capitalism that grew and matured in its own country, namely Europe and North America.\vskip 0.2in

\subsection{A Young Capitalism}

Because capitalism in Indonesia is still young, its production and centralization 
have not yet reached the proper level. About a quarter of a century ago, industrialization 
began in Indonesia. It was only at that time that modern machines were used in sugar, rubber, tea, oil, coal and tin companies.\vskip 0.2in

Indonesian industry, especially the agricultural industry, remains limited 
in Java and in some places in Sumatra. Vast lands, which are usually very 
fertile and contain priceless metal goods, such as Sumatra, Kalimantan, 
Sulawesi and other islands are still waiting for development. Although 
the island of Java in terms of plantations and means of transport has 
reached a high level but generally the islands outside of Java, except Sumatra, is still a jungle.\vskip 0.2in

Real modern industry will not be held on the island of Java. It will remain 
a place for the agricultural industry. Because metals such as iron, charcoal, 
kerosene, gold and others, are not found there or very little. Sumatra is 
the real place of modern industry. This has now been partially proven. Charcoal, 
kerosene, gold and tin from Sumatra (later also iron) are of great significance, both nationally and internationally.\vskip 0.2in

Britain, the world's oldest industrialised country, in the middle of the last 
century underwent a drastic change in its industry. Other European states and 
North America followed gradually. The techniques and regulations working there 
have now reached a level as high as never before known in the history of the world. 
Production and distribution power far exceeds the limits of national needs. Europe 
and North America have become mature capitalist states.\vskip 0.2in

Capital separates the city from the village. Cities produce industrial production 
and agricultural production. The more advanced capitalism is, the more people who 
used to be in the villages are drawn to the cities. When the political and 
economic conditions are well, don't we get more jobs, more institutions of 
education, and more entertainment in cities than in the villages? In 1790 in the 
cities lived 3.4\% and in the villages 96.6\% of the total population, and in 1920 it 
became 51\% and 49\%, respectively. In 1870 the figures were 21\% and 79\% and in 1910 
they were 51\% and 49\%, respectively. Thus, the population in the villages in 1920 was 
smaller than the population in the cities. These figures show us clearly the progress of 
American cities, as a result of the progress of industrialization. In England 
the process of division (of towns and villages) is equally orderly and equally 
adequate. In 1850 the cities lived 49\% of the total population. 
In 1900 this ratio became 77\% and 23\%, (The relations Government to industry, M.L. Regua).\vskip 0.2in

According to foods No. 73 this year, the number of residents and 
cities with more than 10,000 inhabitants in Java and Madura is only 60\% of the total population.\vskip 0.2in

If we use the comparison between urban and rural population as a measure of 
industrial progress of a country, surely Indonesian industry is still in its infancy.\vskip 0.2in

If we also take the total length of the railway to describe the progress of 
industry as an explanation of our description above, it is clear to us that Germany, 
with 177,000 square miles in area and a population less than Indonesia, in 1913 had 
38,809 miles of railway. , while Indonesia, which covers an area of 735,000 square miles, in 1919 only had 3,914 miles.\vskip 0.2in

Regarding the amount of trade (import-export) in Indonesia in 1924 
(after the world war) there was f 2,208,800 (according to International Ocean, 
no. 526, Germany in 1913 [before the war] there was f 13,375,000,000). 
These numbers show our backwardness. But when compared to countries like the 
United Kingdom, India, and the Philippines, it seems that Indonesia is not 
that backward. And when compared to Turkey, Siam, and China, Indonesia 
is much better. By making that comparison as we have already done, in fact 
this has gone beyond necessity. Our purpose is nothing but to explain how young capitalism is in Indonesia.\vskip 0.2in

\subsection{Improper Growth}

Capitalism in Indonesia was not born out of the indigenous ways of production 
according to the will of nature. It is a foreign tool used for foreign interests 
that violently pushes against the indigenous system of production.\vskip 0.2in

When we look at the development of capitalism in Europe and America, 
we realise that older modes of production are successively replaced by 
younger ones. Usually this is not so apparent, but sometimes it is so 
rapid that it is quite obvious. The most recent occurrence is due to new incomes. 
Whatever the circumstances of the moment, it is progress according to nature, because 
the forces driving it are within the grasp of the peoples of Europe and America themselves.\vskip 0.2in

As we have shown, industrial progress in every country is paralleled by the 
rise of cities that manufacture mainly industrial goods such as ironworks, 
agricultural tools, medicines and so on. The villages produce rice, vegetables, 
livestock, milk and so on. Excess city goods - that is, goods that the city dwellers 
see as necessities of life are exchanged for excess village goods.\vskip 0.2in

In America at a typical period like 1913, when the country was remote and less 
imperialistic, as it is today, the ratio between industrial and agricultural goods 
was practically the same (the market price of both goods was almost the same). 
So in economic terms, the city fulfils the needs of the countryside, and the countryside fulfils the needs of the city.\vskip 0.2in

In Indonesia, as a result of disorganised economic progress, the situation is not as it should be. 
Our cities cannot be regarded as concentrations of technique, industry, and population. 
They do not manufacture goods either for the countryside or for foreign trade, from indigenous capitalists. 
Agricultural machinery, household utensils, materials for clothing etc. are not made in Indonesia, 
but are imported from abroad by imperialistic trading agencies. Our villages do not produce the goods 
needed for the cities, because they do not even have enough for themselves. Rice, for example, the main 
food of the people, has to be imported, in 1921 at a cost of f 114,160,000, although our people are 
generally very good at working their land and have all the conditions to produce rice for their own needs 
and can even produce excess rice. Our villages produce sugar, rubber, tea, and other trade goods that enrich 
foreign merchants, but impoverish and destitute the poor; our cities are not the economic centres of the 
Indonesian nation, but continue to be economic sources of profit for foreign financial vultures.\vskip 0.2in

The ingredients that make capitalism not Indonesian - given the history of our country mentioned above - are clear to us.\vskip 0.2in

We have seen that the robber politics of the Dutch destroyed all the seeds of modern native industry. 
\emph{cultuurstelsel}\footnote[4]{A Dutch government policy from 1830–1870 for its Dutch East Indies colony. Requiring 20\% of agricultural production in all villages to be devoted to export crops (sugar, tea, coffee, etc) and for people that couldn't meet the requirements, they were required to work for 75 days a year in a government owned plantation to make up for it.}, 
monopoly landlords and unforgiving taxation. And the regular influx of Chinese merchants in 
the days of the Far East Company (VOC) destroyed all strong national socio-economic and technical means.\vskip 0.2in

If the Indonesians had not been robbed, and had the technical intelligence 
and influence of foreigners, they would have had the opportunity to fulfill the will of nature.\vskip 0.2in

Either peacefully (as in Japan) or by means of a national boycott (as in India) 
the Indonesian or Indo middle class by pooling national capital set up industries to fulfil national needs such as iron workings.\vskip 0.2in

Thus, Indonesian capital arises in an orderly manner among the social strata 
of Indonesia and has an orderly relationship. Indonesian merchants who were once 
small are now bankers or heads of large companies. The once small iron forge, sugar 
artisan, batik merchant became the leader of the metal, sugar or weaving industries. 
But in the 300 years of Dutch imperialism, nothing has improved for the Indonesian 
people, everything has been transported to its own country. It unleashed out Dutch colonial capitalism like no other in the world.\vskip 0.2in

To enter into an economic struggle against foreign titans of industries, 
with the intention of boosting national industry, is to "catch the wind".\vskip 0.2in

\subsection{Indonesian Capital is International}

British imperialism with its first-rate national industry and a formidable fleet, had from 
the beginning felt the need to compromise with the kings, and landlords of the Indian nation, 
to defend itself against the nascent native bourgeoisie. But when the latter came out of the 
struggle victorious (in 1900-1905 and 1919-1922), the British had their hands full.\vskip 0.2in

Together with the kings, landlords and the new Indian bourgeoisie, she went to 
ride on the backs of the grumbling people. However challenging British imperialism was,
it still had a purpose within its own empire.\vskip 0.2in

Dutch imperialism punched and kicked the patient "buffalo" for so long that it is now using its horns.\vskip 0.2in

The little Netherlands, which in the past swallowed everything for itself, is now forced to share it with the stronger countries.\vskip 0.2in

As the lack of capital and industry was the most important cause of these Dutch actions, 
since a few years ago, British capital has played a major role in Indonesia. 
The wise Raffles had long seen this and was not satisfied until he could trick the Dutch-peasant's eye. 
After the war with Napoleon ceased, the British returned all the Dutch colonies. This action seemed to be very 
contrary to the politics then used by the British, but upon closer examination it was the British policy of being 
as subtle and cheap as possible in using the Netherlands as an instrument for the capital they had invested in 
Indonesia. Did the takeover of the entire Indonesian administration give the British responsibility and trouble? 
The British capital, which in recent years has been getting bigger and bigger, for the Netherlands - 
is becoming alarmingly small, and the Indonesians are now so impatient that the Netherlands now intends 
to use "open-door policy". This term, which is actually taken from an American dictionary, fits perfectly 
with Dutch politics in the East. In ordinary words, it reads: "And against British capital and the Indonesian 
nation, which has woken up from its slumber, the Netherlands should be stronger if it has a democratic America. 
But this country must be drawn to Indonesia. Its capital must be invested in Indonesia with all its endeavours and, 
if necessary, given extraordinary rights. When the time comes, America will join hands with the Netherlands".\vskip 0.2in

Money and labour no longer count in favour of American capital. 
A minister once said frankly in the chamber that: The arrival of American capital is very easy because of 
the present laws in Indonesia. Fock's visit to Manila in 1923, and the arrival of several warships to the
Philippines, placed a consul-general in New York whose job was not only to go back and forth with negotiations 
and agreements but also to squander money on billboards, pamphlets and magazines that for years carried the story 
of Java the Wonderland. All this was to lure American travellers and capitalists to Indonesia.\vskip 0.2in

The size of the Dutch capital can be seen in the figures below.\vskip 0.2in

In the Handbook \emph{voor cultuur en handsondernemingen} in Ned, India, written by Agulvant, the capital 
invested in Indonesia is estimated at f 3,270,000,000. Of which f 1,27,000,000 in plantations, f 900,000,000 in oil. 
In banks and trade f 750,000,000.\vskip 0.2in

Ship, railway and tram companies f 250,000,000, f 220,000,000 and f 200,000,000 respectively. 
Mines f 70,000,000 and insurance companies f 60,000,000.\vskip 0.2in

Capital invested in East Sumatra in 1924 totalled f 439,000,000. Of which 55.3\% belonged to the Dutch and 44.7\% to foreigners. 
Foreign capital invested in the agricultural industry totalled f 200,000,000. Of which f 147,500,000 is British capital, 
f 300,000,000 belongs to France and Belgium, f 15,700,000 to Japan and f 4,000,000 to Germany \emph{(International Ocean. No. 6, 1926)}.\vskip 0.2in

The area of rubber plantations in 1924 was 241,357 bau\footnote[5]{1 bau = 500 tombak persegi or 7096 m2}. Of which 42.\% belonged to foreigners and 32.\% to 
the British. Due to the British monopoly, American rubber capital has been rapidly increasing in Sumatra in 
recent years. The area of tea plantations in Java is 116,664 bau. Foreign ownership 23.\% and British ownership 17.\%.

Of the seven main products shipped to markets around the world, sugar exports in 1924, f 491,100,000 or 32.\% 
of total exports. Rubber f 202,600,000, or 13.2 per cent of exports. Kerosene f 158,300,000, tobacco f 123,600,000, 
copra f 97,400,000, tea f 93,600,000 and coffee f 56,600,000 i.e. 10.\%; 8.\%; 6.\%; 6.\%; and 4.\% of total exports respectively.\vskip 0.2in

In 1924, exports to the United Kingdom and its colonies accounted for 42.55% of all exports, while only 19.7% went to the Netherlands and 40.4% to the United Kingdom and its colonies.

So it is clear that British trade in Indonesia is larger than in all foreign countries, while in the most important oil companies and plantations, British capital plays the largest role among non-Dutch capital. So it is not surprising why the Dutch are in a hurry to lure American capital.

It is true that in recent years, jealous of the British monopoly on rubber, 
the Americans have begun to invest in rubber plantations in East Sumatra. However, 
it is not yet certain whether America wants to invest its capital in Sumatra and Java 
alone, because there is fertile land for rubber plantations in Mindanau (Southern Philippines) and Liberia.\vskip 0.2in

Recognising and safeguarding modern native industry as in India according 
to the new economic outlook will not exist at all, because modern native 
industry does not exist. The people are only squeezed, trampled and cheated. 
The dismissals of labourers are not unusual, and the grip of taxation is 
getting tighter and tighter. The people's economy need not be mentioned because 
the Netherlands is mainly dependent on foreign capital.\vskip 0.2in