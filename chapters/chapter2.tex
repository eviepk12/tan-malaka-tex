\section{Foreign Influence}

The history of Indonesia is not easy to read, let alone write. The history of 
our nation is full of mysticism, legends, stories, and contradictions. No historian 
from the Majapahit Empire or Mataram has come close to the historians of Rome more 
or less 1400 years ago, like Tacitus or Caesar. We are forced to admit that we have never known an honest historian.\nline

At best we have fairy tales, legends, and bootlickers of the king that tell all sorts of 
beauty and brilliance to attract the listener's heart.\nline

Even then there is a limit to stories and twists of real events. We don't have to go far to 
the limit, surely we will meet the truth. So is the history of our nation. Among the clutter 
of stories, we can see the truth, the Indonesian archipelago, its kingdoms, cities that 
rose and fell, the armies that roared, fought, lost, and won, wealth, peace, the ups and 
downs of our culture, and so on. It's undeniable that in Malacca, Sumatra, and Java stood 
great nations. In Central Borneo stood kingdoms that are no less prosperous and impressive 
than the Majapahit. There stood great cities full of buildings and beautiful jewels, as 
evidenced by the objects found in the fossils and ancient temples up to today.\nline

We can also confirm that Indonesia never stepped out of its feudal society, and one that is 
far removed from feudalism in Europe. Greek society is far above Indonesian society - in this case, 
if the Majapahit is considered as the highest level - in terms of state government, politics, 
law, and culture. Yes, the people of Majapahit have never known the ideals of state government. 
For centuries the government was not for or owned by the people, a proverb: \emph{“For my sir, yes, my lord, independence, mine and my life,”} 
once and repeatedly said the Indonesian people to their kings! There is no Orachus, Magna Charta, 
and no knowledge that the likes of Aristotle, Pythagoras, and Photomeus used to advance their societies. 
The knowledge of architecture and medicine for us is still in the stage of experimentation. 
The miracle of our Borobudur is not as miraculous as the triangle of Pythagoras, because the former 
means the path of death while the latter leads people to various kinds of knowledge. Nowhere are such 
traces of knowledge and pinnacle of intelligence!\nline

That said, let's not talk about Eastern Mysticism! This is beyond our material knowledge; moreover, 
the Western nations in the Dark Ages (Middle Ages) have known it for long. After all, mysticism is 
not based on material truth, our society has always obtained from foreign influences and never has 
its own ideals. Hinduism, Buddhism, and Islam are imported goods, not the product of our society.\nline

In addition, these ideals did not grow as much as Christianity in Western Europe. 
The driving force behind the entry of Hinduism, Buddhism, and Islam up to the time of 
the arrival of Dutch capitalism, and all civil wars at that time were abroad. Indonesia 
had always been the puppet, and the masters are always abroad.

\section{The Original Indonesian Nation}

In ancient times, when the indigenous Indonesians were 
pressured by the Chinese and Hindus coming from their land - the Indies - and 
fled to the Indonesian Archipelago, they already had established civilization. 
The peasants of the time turned into pirates who were ferocious and feared by the people. 
With their Vintas (a small type of boat), they sailed across the islands between two large 
oceans, between America and Africa. The Indians and Oceania were conquered by them. The jungle 
to the top of the mountain was turned into their new home. Those homes they made, games, 
and knowledge. When the Western and Eastern nations worshipped the sword of Genghis Khan and 
Timur Lang and fled in fear, at that time they not only resisted but were also able to repel 
the Mongol army. The pirates named Pakodato from the Kingdom of Singapore in the Malay peninsula 
500 years ago were able to shake the Kingdoms of China and India with their mighty fleets and swords.

\section{Hindu Influence}

In the strong and famous Kingdom of Daha ruled by King Jayabaya, a smart and clever, 
even yet wise, there was a fortune teller named Empu, who was always upset because he 
was very suspicious of foreign influences that were getting relevant and relevant. 
In his writings, it mentioned: “A revolution in Java will arise, led 
by yellow people and will gain victory for some time”. In his sarcastic 
words, it's written “it will reign for a lifetime”.\nline

Isn't that prediction then proven by the victory of a Chinese Javanese 
named Mas Garendi that for a short time controlled the city of Kartasura?.\nline

In the time of Empu Sedah, the influence of China gradually increased.
It is logical for Chinese people to use the Javanese nobility as a tool to fulfil 
their economic interests as much as possible!\nline

If this goal does not succeed with their influence, sometimes by way of revolution then 
try to seize the state government. But, in order for them to keep their power, they must be 
stronger or establish their own class. They must be natives or inter marry with the indigenous 
people. Only then could they conquer the kingdom through the displeased peasants. Because the 
Chinese socially remained with their Chinese heritage and did not receive military assistance 
from their motherland, they were soon able to maintain the victory of the Javanese kings.\nline

Empu Sedah understood how much hatred the people and the revolution had if it were to break 
out. While their power nationally is not strong enough to withstand the social revolution. That's what made him nervous.\nline

In the Majapahit Kingdom stood several batik, tiles, and ship companies with considerable capital. 
Some companies employed several thousand workers. Some of its captains had sailed as far as Persia 
and China with their ships. It can be because they had such a big capital, even foreign capital. 
The rich merchants in cities such as Gampel, Gresik, Tuban, Lasem, Demak, and Cirebon were probably 
foreigners or who had intermarried with the Javanese. Nakhoda Dampu-Awang, according to his exaggerated 
story, had a ship whose sail was as high as Mount Bonang, and his wealth described as such, they feel like Chinese-Javanese.\nline

One statistic at that time we did not have! But the nationality of peoples living on the island of Java can 
be proven by the words of a Majapahit poet, named Prapanca, “consistently people came in droves from various nations. 
From the Mainland-Indies, Cambodia, China, Annam, Champa, Karnataka, Guda, and Siam with ships accompanied by many merchants of 
renowned Brahmana clerics, scholars, and priests, ready to come to be entertained and to stay”\nline

Of course, the inhabitants of the increasingly developed cities felt they were gaining resistance to the nobility in 
the capital. As in European countries, urban dwellers were demanding further political and economic rights. From 
the contradiction between the coast and the land, trade, and agriculture, the population and the government, 
a revolution arose that brought the island of Java to the pinnacle of economy and government.\nline

If the city had a strong national industry and trade, Java would surely experience a social revolution that is 
raised, broken, and led by a social revolution that is raised, broken and led by national progressive forces 
as happened in Western Europe, the bourgeois revolution against the feudalists.
But Java was indeed bound by Empu Sedah's prophecy: “foreigners will rule”.\nline

A Hindu named Malik Ibrahim in 1419, brought a new religion not yet known in the Island of 
Java, came to Gresik where at that time the residents were mostly foreigners. Quickly he 
gained a number of followers. So it can be said, with the arrival of Islam at that time, the 
indigenous people as if gaining “the fallen durian”, because at that time there was a fiery 
conflict between the coastal population and the capital.\nline

The situation became more complicated, and eventually reached its peak, 
the attack on the king led by a Chinese-Javanese, named Raden Patah. By his actions, 
Raden Patah destroyed the existing kingdom. It shows again that a foreigner, 
by bringing a new understanding (Islam) and to maintain the position of foreign merchants on the c
oast, succeeded in overthrowing the half-hindu aristocratic kingdom. The kingdom of 
Demak stands by its fame! But it was eventually torn apart by a merchant war ignited by cunning foreigners.\nline

Jipang was hostile to Pajang, Demak to Mataram. All these civil wars, 
large or small, for the benefit of foreign nations, in a short time ended with the victory 
of a Chinese-Javanese named Mas Garendi.\nline

\section{Tarunajaya}

As in the Roman Empire and China, the mounds of government officials who did not match 
the truth in the capital were swept away by new powers from the regions; thus, the blood 
of the Mataram Kingdom will be cleansed and strengthened by Tarunajaya and his friends.\nline

An Indonesian prince who came from Makassar knew the soul of the Javanese people got a large 
following, and managed to defeat the King of Mataram who came out of the true line. Java in 
particular and Indonesia in general will have a different history if a new power does not come 
to Java. Empu Sedah's other prophecy now seems to be proven, "The government of a foreign nation, 
namely a white buffalo with eyes like a cat's eyes" (kebo bule siwer matane).\nline

With the advent of Dutch rule, everything resembling independence disappeared. The influence of 
foreigners and the mixing of blood with other Asian nations caused a great deal of tension. All 
economic and political rights were "swallowed" by the Netherlands with violence and fraud, as the 
Indonesian nation has never known! Extortion of the lowest form (barbarism) and tyranny have become a daily habit!\nline

Tarunajaya could not resist the Dutch rule using foreign weapons. So the cat saw this situation and 
for the first time used the political path of devide et impera, splitting and dominating that worked 
wonderfully for the colonialist. After the King of Mataram promised the East India Company to give 
them power and land, the demons began to work.\nline

Panembahan in Madura, a friend from Tarunajaya, was stuffed by the East India Company with diamonds 
and sweet words until they could get along. Now Tarunajaya stands between the “three fires”: the Dutch, 
the king, and old friends. This is what caused the defeat of Tarunajaya witnessed by the East India Company itself!.\nline

The indiscriminate Mataram government received a "victory" due to the indirect support of the Company, but 
something indiscriminate will sooner or later become a reality as well as proven in the end.\nline

\section{Diponegoro}

The road from Anyer to Banyuwangi, which connects the robbed districts, was built by Governor General 
Daendels with a lot of sweat and the lives of Javanese people at stake. With that road, 
the process of capital cultivation becomes orderly. But the process was not voluntarily accepted by 
the Indonesian nation. It is a coercive process and not according to the laws of nature. Merchants in 
the cities were urged. Sailing was monopolised by the Dutch, indigenous people were forbidden to 
have property rights. Cheap cotton imports from the West destroyed industry and trade, both 
small and medium. The Javanese bourgeoisie or half of Java can continue its journey, namely 
the journey between feudalism to capitalism. However, it was squeezed to dryness by Western 
capital and its apparatus; such is the feudalism of Mataram that almost fell.\nline

A boy with a will as hard as steel, as influential as a hot iron, that is, a man who has 
in his chest the qualities of a true Indonesian prince, powerless to change that 
unfortunate fate. If Diponegoro was born in the West and placed himself in the face of 
a revolution with such a sacred heart, it may be that he will be able to match Cromwell 
or Garibaldi. But he “helps a leaking boat”, a class that will disappear. His deeds, 
though full of chivalry, in the economic view were counter-revolutionary. And it is 
very difficult to ascertain, what kind of Diponegoro in the political view, because it 
is undeniable that his ideal is "Singgasana Kerajaan Mataram"(Throne of the Mataram Kingdom). 
A power that can easily turn into tyranny.\nline

Diponegoro supports further capital expansion as well as road expansion. Therefore, 
it hinders income and economic growth, counterrevolutionary. We never read that he 
opposed capital-imperialism by reviving national capital. In short, he has no political 
or economic program. He felt pressured by the new power and after he saw that the new power 
was using the decayed power of Mataram as a tool, so he attacked the two enemies.\nline

If Java had a revolutionary national bourgeoisie, Diponegoro in his struggle against 
Mataram and the Company must have stood on the side of the bourgeoisie. In this way, 
a noble and certain deed can be created. But that was not the case, the bourgeoisie 
that smelled of Islam in the economic field was destroyed by Dutch capital altogether. 
In great frustration with Mataram and the Company, it was able to unite under the 
leadership of Kyai Mojo, a fanatical Muslim cleric with the slogan "War of Sabilullah", not national.\nline

Drawing a conclusion on the Diponegoro uprising is not an easy job. Because of this, 
the struggle of the Javanese Islamic bourgeoisie against the Western capital is supported 
by a kingdom that is about to fall (Mataram).\nline

The consequences are very clear. No one is able, however smart, to help 
a weakening class, both technically and economically, against a class that is getting stronger and stronger.\nline

A new class must be established in Indonesia to fight modern Western imperialism.\nline

What are the conclusions from the above stories?\nline

First, that our history is Hindu history or half Hindu history; second 
that the feeling of national grandeur is far from where it's supposed to be; and 
lastly, that every thought that aspires for development (renaissance) has the same 
meaning as digging into the aristocracy and colonisation of the Hindu nation and the half-buried Hindu nation.\nline

The true Indonesian nation from then until now still remains an obedient slave from foreign robbers.\nline

The true nationality of Indonesia does not exist unless there is an intention to liberate the Indonesian nation that has never been independent.\nline

The true Indonesian nation has no history of its own other than slavery.\nline

The history of the Indonesian nation will only begin if they are free from the oppression of the imperialists.