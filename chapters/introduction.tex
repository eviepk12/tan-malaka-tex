\begin{center}
    \textbf{Alles was besteht ist wert,\\
    dass es zu Gruende geht.\\
    (Mephistopheles)}\nline
\end{center}

Asia has awakened! \nline

Sooner or later, the enslaved nations of Asia will surely gain freedom and independence. 
But not a single person can say when or where the first flag of independence will fly. 
Whoever investigates deeply into the economies of the east, politics, and sociology will 
be able to point out the weakest link in the long chain that binds the east in slavery. Indonesia is that weakest link. 
In Indonesia, the first bastion of Western Imperialism can be fought successfully. \nline

Dutch imperialism is older than British and American imperialism, divided by the great valley that could not be crossed from its colonies. 
The Netherlands, because it does not have resources for its industries, from the beginning only focused on agriculture and trade\nline

The spread of their capital from the beginning of the century to all of Indonesia are enormous\nline

The industrial centre of the Netherlands now is located in Indonesia, while the centre of trade and finance is in the Netherlands. 
Bankers, industrialists, and merchants live in The Netherlands, while the workers and peasants in Indonesia. 
If we pay attention to the 2 oceans that divide The Netherlands and Indonesia, while also not forgetting the differences of 
culture, religion, tradition, and language between the colonisers and the colonised, between the exploiters and the exploited,
we see a remarkable comparison in the world of imperialism which we live in today. Remarkable, because the indigenous capital 
does not exist. So the bridge between The Netherlands and Indonesia breaks completely\nline

The lack of a national bourgeoisie that have the same characteristics as Dutch imperialism 
(both want to exploit the workers and peasants) caused Dutch imperialism to find it 
very difficult to resolve the economic crisis in Indonesia. Where in Indonesia are the 
indigenous landlords like in Egypt, India, and The Philippines, that can support the imperialists 
for the sake of their economic interest? And where are the strong national bourgeoisie that ask 
for more political and economic power like in India\nline

The lack of Indonesian capitalists means for long they have been slaves! 
Europeans, Chinese, and Arabs have controlled all big, middle, even small trades! The medium 
or small Indonesian producers have withered away from the Island of Java for the last few 
years from the introduction of European industrial products\nline

The case of education is intentionally ignored by the Dutch, the intellectual class 
became small. Even if the national bourgeoisie like in India wants to support them 
to establish an industry, it will not succeed\nline

The lack of a national bourgeoisie makes all parliamentary actions from a national party useless\nline

How will the “sugar daddies” in The Netherlands be able 
to give the Indonesian people the right to vote? Or in other words: entrusting political 
power to the poor workers and peasant supposed representatives? If behind the intellectuals, 
stand the landlords and the national bourgeoisie that they will represent in parliament, 
of course, the situation will be different. And rumours about “changes in the government in Indonesia” 
do have some merit. The Dutch imperialist gradually, but slowly will hand over governmental authority 
to the Indonesian people that are honest and competent. Isn't protecting the national bourgeoisie, 
also means protecting foreign capital? Under the current situation, every type of government in 
Indonesia has to be subjected to the whims of foreign capital. And a government like that will never be recognized as, 
from the people and by the people\nline

In short, Indonesia has no economic, social, or intellectual factors to liberate itself 
from economic and political slavery within the sphere of Dutch imperialism. At the same 
time, the chances of achieving independence in the fullest sense by controlling half, 
three-quarters, or seven-eight of parliament is impossible. The dream of Noto Suroto\footnote[1]{Noto suroto was one of the earlier Indonesian intellectuals, being one of the founding member the Indische Vereeniging and was a supporter of integrating Indonesia into the Netherlands}
and the likes that dream of “The Great Netherland” will remain the daydreams of the wicked\nline

Indonesia can only raise its economy if political power lies in the hands of the people. 
And Indonesia will only gain political power only through well-organised revolutionaries, who does not seek compromises\nline

The People's Council\emph{Volksraad} can sometimes be entered! But it is not to be used as a legitimate weapon 
to obtain a fully responsible national government by means of the People's Council in 
co-operation with the Dutch imperialists. But to extend the revolutionary effort into 
the chambers obtained by means of parliamentary actions is akin to someone in the Sahara Desert 
chasing apparitions. But whoever puts all his knowledge to use in organised mass action will win 
the victory like a "chicken coming home to roost"\nline

The matter of independence is not limited to just Indonesia, which can be solved through congressional meetings 
in the people's council, not to mention the discussions of economic and cultural jokes in the coffee shops. 
Such matters are closely interlinked with western hegemony towards the people of colour in Asia\nline

One of the reasons - and this is not the least of the problems - why the United States has not also granted independence 
to the Northern Indonesians (the Philippines) who, according to the words of friends and foes, have long been ripe 
(as the American Imperialist newspapers in Manila say) is that Philippine independence would mean a general revolt and slaughter in 
Asia against white authority (a general revolt in Asiatic countries against white authority, the uprising being attended by slaughter). 
The independence of Indonesia (the centre of Asian geography and warfare, five times more populous than the Philippines 
and with international trade) could only mean like a pistol aimed at the heads of western, especially British, power in Asia\nline

Not long ago the former crown prince Willhelm explained to a representative from the United press 
in Locarno which announced by radio to the world, that if the millions of people in Asia one day beat 
the Anglo-Saxon (British, French, and Dutch) surely it would be the nations of Malaya who first caused it. 
What kind of imperialist expectation and insinuation were meant by the irritating crown prince, for us 
it remains to mean: that Indonesia now is not Indonesia from a few years ago. Indonesia has taken a special 
position in the ranks of millions of people in Asia\nline

Therefore, the victory obtained by peaceful and parliamentary means should not be thought of at all. 
Wouldn't such a thing not exactly disturb the capitalist tranquillity in the East? If Indonesia one day 
breaks free and defends its independence from its domestic and foreign enemies, it will be determined by its 
revolutionary nature, that is, by mass action: from the masses and for the masses\nline

If the 300 years of Dutch colonialism were not in the form of robbery (killing indigenous industry) surely 
the degree of intellectuals would be far different from where we are today! And we would of course have 
the intellectual fervour (intelligentsia) according to our origins, education, and loyalty to the leaders 
of the national bourgeoisie, industry, merchants, and indigenous officials. There will also be a movement 
for democracy and national liberation that is cooperative (compromising) with The Netherlands 
with the help of workers and peasants as in India, Egypt, and in some ways The Philippines\nline

The lack of national bourgeoisie makes the intelligentsia weak. They hover between the people with the government. 
They don't have a sense of self-sacrifice like other nationalists have shown in other countries. They don't have the 
necessary connection, way of thought to bring themselves closer to the masses. Caused by imperialists, they made our class 
of intellectuals disconnected from the masses\nline

They do not have the necessary power to influence and attract the hearts of the people. Our 
intellectual class doesn't have the trust and sympathy from the masses to mobilise them, organise actions, 
and lead them. In addition, due to the small number of intellectuals, they still live in their own classes and 
have not become educated workers\nline

For the time being, they can watch from afar. If, on the other hand, there were many of them, they would be 
roaming around and experiencing the misery of industrial labour with the "joy" of the struggle.\nline

The speed in which the intellectual class emerged, the disillusionment of Budi Utomo (B.U)\footnote[2]{Budi Utomo was one of the early indigenous organisation mainly consisting of Javanese intellectuals and aristocrats}
and the National Indische Party (NIP),\footnote[3]{The NIP was an organization and noted for its focus on political work and support for independence}
and the brutality of reaction, have clawed away their sights to another direction. Although now still stands a few miles apart 
from the masses and political participation, politics is very much backward compared to comparable colonies of other nations, 
but they have surely awakened from their slumber. “The cloak of an angel” from Noto Suroto has been tossed away by them and 
starts to go in the path of revolutionary actions. Now from several universities in the distant Netherlands, their voices 
echoed all the way to the intellectuals in Indonesia\nline

But the hope of the workers and peasants of Indonesia does not just stop at agreements in words from the intellectuals. 
They demand actions and evidence\nline

As long as our intellectual class still sees our struggle for independence as nothing more than an academic matter, 
such promises and demands for action are nothing more than empty promises. Let them step out of their study rooms and immerse themselves in active revolutionary politics\nline

The wave of strikes, boycotts, and demonstrations grow larger and larger every year, through national congresses 
towards the Federation of the Republic of Indonesia, this is their way, nothing else!\nline

\begin{flushright}
    Tan Malaka
\end{flushright}
