\section{The High Probability of Revolution}

There is no need to elaborate on the political, economic and social issues 
that might lead to revolution in Indonesia, as we have already explained 
them several times above. It is enough to draw the following conclusions.\nline

\begin{enumerate}
    \item Wealth and power have accumulated in the hands of a few capitalists.
    \item The Indonesian people are all getting poorer, destitute, oppressed and subjugated.
    \item Class and national contradictions are getting sharper and sharper.
    \item The Dutch government is becoming more and more reactionary.
    \item The Indonesian nation is becoming more revolutionary by the day and knows no peace.
\end{enumerate}

Since the notion that the Dutch imperialists will suddenly become intelligent, ingenious and able to make 
compromises to the detriment of the big capitalists can be regarded as a fantasy in "The Story of a Thousand and One Nights", 
the revolutionary development will not be stifled. On the contrary, it is progressing faster and faster 
and an outbreak of revolution can be expected at any time.\nline

Moreover, some of these revolutions have already been proven. The several rebellions that broke out on 
their own in Java and Sumatra during the 300 years of the "blessed" Dutch imperialism were the result 
of a clash of classes and nationalities that had originally taken the form of religious rebellions. 
Also the political turmoil of the past 15 years, in the form of various incitements and actions and 
more clearly in the form of anarchist plots and actions in Java and the murder of Pamong Praja officials 
in West Sumatra which undermined the confidence in the invulnerability of Dutch imperialism, are all 
classified as the result of class and nationality conflicts.\nline

However, a major clash between classes and nationalities of such a magnitude, breaking out solely 
because of the clash itself and of such a modern nature as a "revolution", has not yet occurred in Indonesia!\nline

Eventually it will engulf the entire archipelago and erupt on its own.\nline

\section{The Emerging Nature of the Indonesian Revolution}

What will the revolution look like? What characteristics will it exhibit if it erupts tomorrow or the day 
after? This is what we, as revolutionaries, must ask ourselves and answer at once, if we are to avoid the 
"wandering" politics of Douwes Dekker and Tjokroaminoto. According to the answer to that question, we forge 
the tools of revolution, namely our organisational programme and tactics.\nline

The proper characterisation of Indonesian society is the most important prerequisite for obtaining the tools of 
revolution. It is also the first step towards the victory of our revolution.\nline

If the analysis is imperfect or we are wrong with our predictions and conclusions, the victory will be 
uncertain or short-lived. We don't have horoscopes that can foresee future events the way an astrologer 
predicts a person's future life. However, with Marx and Lenin as guideposts, we can determine some of the 
outlines of the revolution in Indonesia (given the current level of capitalist development).\nline

Certainly the revolution will be different from the "Moroccan Rebellion"\footnote[10]{Referring to the Rift War between berber tribes of northern Morocco and Spain from 1921-1926, joined by France in 1924.}. This is particularly true because 
Indonesia has a higher level of productive power (industry, agriculture, transport and a large financial base) 
than the small peasant and sheep-herding country of Morocco. Also Indonesia, especially Java, does not have inhabitable 
mountains and vast deserts where revolutionaries can hide for years and then at any time resume guerrilla warfare.\nline

Furthermore, it will not be a genuine proletarian revolution like in Germany, Britain and America 
(where the population is mainly made up of proletarian workers) because Indonesian capital is still too young, infertile and weak. 
Therefore, our workers, when compared to the workers of the West, are far behind, both in quantity and quality. In addition, 
the conditions of the non-workers who will also participate in the revolution are still within the epoch of the bourgeoisie 
revolution and the national revolution.\nline

Our revolution will also not be like the bourgeois revolution in France in 1789 because our bourgeoisie is still too 
weak and feudalism has been largely wiped out by Dutch imperialism. Nor will it equal the French Revolution of 1870 
because we seem to have a more developed production forces, in addition to an excessive social contradictions.\nline

It will also be different from the Russian Revolution whose feudalism can be said to be weak and whose young bourgeoisie 
has been greatly weakened by years of war, while the workers are young, happy and educated according to Lenin's principles. 
We must fight against Western imperialism even though it is small, it must not be ignored because it has tricks and likes 
to be the "servant" of mighty British imperialism.\nline

In the end, it will not be a purely political revolution like the ones in India, Egypt and the Philippines, where the 
native bourgeoisie only seized political power (parliamentary power) because the national capitalists are strong and the 
intellectuals are more numerous than in Indonesia.\nline

The Indonesian Revolution will be partly against the remnants of feudalism and partly against despotic Western imperialism. 
It is fuelled by the hatred of the Eastern nations for the Westerners who oppressed and desecrated them.\nline

The heart of the revolution (at least in Java) must be formed by the workers of modern industry, enterprise and agriculture 
(factory and farm workers). The political bastions, especially the economy of Dutch imperialism, can only be beaten by the workers. 
Surrounding the workers are the petty bourgeoisie, who will go backwards and forwards (The bourgeoisie will obey if they know they 
will win; even then, they will go backwards. Even if they really like to participate. Anything more than that is a "no" and should 
not be expected).\nline

A victorious Indonesian Revolution will bring about the right economic, political and social changes at a time when capitalistic 
development is facing a crisis. If our workers remain active, they can play the most important role.\nline