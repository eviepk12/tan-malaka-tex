\section{In Retrospect}

"Politics" in Indonesia has never been "a common good", the common property of the people. 
The notion of statehood has never gone beyond a small clique of Hindu or half-Hindu colonisers.\nline

As in most feudalistic societies in Indonesia, the government of the country is held by a 
king and his accomplices. The king, having successfully performed the role of the "hero", 
then appointed himself as the reigning king. His son, who could be dumber than a buffalo, would 
later replace his father as the lord of the land. This hereditary rule "vanishes" when a 
new "hero" comes along and overthrows the old one, making himself king of the land.\nline

No constitution determines the coronation or impeachment of a king and his ministers, and specifies 
them carefully. All power and its sphere of influence rests on the king's violence and will, as well 
as the trust and servitude of the masses. Government of the people, for the people, by the people, as 
Lincoln said, has never been known in Indonesia.\nline

Occasionally there is a king who is "fair" on the political stage. However, this is 
an exception, a fluke and an anomaly. There is nothing the people can do in the absence of such 
a king but to revolt. Indonesia has only known the rule of a few and has never known written laws.\nline

The situation in Minangkabau\footnote[7]{Minangkabau is an ethnic group residing in the highlands of West Sumatera.}
is slightly different. Government according to custom is entrusted 
to the representatives of the people, the datuk-datuk, the penghulu. They must rule according to 
certain laws. The supreme power is called \emph{"mufakat"}\footnote[8]{"Mufakat" is a traditional Indonesian term for "to reach a consensus".}, 
which is obtained by legislation in a meeting.\nline

Every meeting must be open to the fullest extent possible and according to established custom. 
Men and women at meetings have the fullest right to speak which shall not in any way be curtailed. 
Whether on local or national matters, "the law" reigns supreme.\nline

However, such a situation only exists in Minangkabau, which is a small, remote area in the Indonesian 
Archipelago. For this reason, the people there were not much influenced by the Hindus and Arabs, in short, in terms of politics.\nline

Despite the Dutch, if they want to treat the Indonesian people with the same respect as their peers, 
for example as in other parts of Indonesia, in drafting and executing laws and in forming and imposing governments, 
"the people must not interfere".\nline

\subsection{Basic Minangkabau Law}

\emph{"Anak kemenakanaja kepada penghulu,\vskip 1mm Penghulu beraja kepada mufakat.\vskip 1mm Mufakat beraja kepada alur dan patut"}.\nline

\emph{(Rougly translates to)}\nline

\emph{"The people follows the leaders,\vskip 1mm Leaders to the consensus,\vskip 1mm Consensus to customs and tradition"}\nline

This was also the case in the kingdoms of Poko-Dato, Srivijaya, Majapahit and Mataram.\nline

Because the people could not intervene in the government of the country, the East India Company was able to 
conquer or compromise with the kings of Indonesia, and gained power little by little, and finally the whole 
of Indonesia fell into its hands.\nline

\subsection{People's Representatives or Soviets}

During the Dutch colonisation, a social discourse was born that gradually 
called for a solution to the question of the organisation of the state but governance was still 
not necessarily parliamentary or Soviet.\nline

Parliamentarianism in Western countries was born by the bourgeoisie when arbitrary rule was 
rampant everywhere and the bourgeoisie, with its advanced commerce and industry, felt that it 
was being hindered in expanding its enterprises by feudal lords: who were hindering it with various 
taxes and fees, while it was denied political rights. It was under such circumstances that 
the Magna Charta, Cromwellism and the French Revolution were born. Voltaire, the great leader of 
the bourgeoisie, then attacked the Catholic religion and its priests and preached "atheism" (denial of God).\nline

Rousseau opposed autocracy with democracy and to oppose hereditary rule, he preached the "social contract", which 
is a government that contracts with the people. According to Rousseau's doctrine, a king should only rule as 
long as he acts in accordance with the contract; the people should oppose him if the contract is broken.\nline

Because the French bourgeoisie felt that it lacked the strength to resist the rule of the king, the nobility and the 
clergy, it united with the revolutionary masses, the workers and peasants. However, these masses were not allowed to 
take power. They were to be used as cannon fodder in the bourgeois revolution, while power was to be held by the 
bourgeoisie. With the slogan of "Liberte, Egalite and Fraternite", now democracy, liberalism and parliamentarism, 
they were able to overthrow the feudalistic government.\nline

After gaining political power, "bourgeois democracy" shows itself. Although in parliamentary states, 
such as England, France and America, every citizen is given the right to vote, the workers and the poor 
there (the largest number of people) are always unable to support their candidates in parliamentary 
elections because they are trapped under the influence of the bourgeois thought developed in schools, 
churches, newspapers, and above all, because they lack the means of propaganda (meeting rooms, newspapers and brochures, 
all of which are expensive).\nline

The bourgeoisie, with its well-paid professors, journalists, priests and diplomats, can win parliamentary elections.\nline

Since members of parliament hold office for three or four years, the relationship between the voter and the elected is 
very tenuous. They deal with the people only at election time, and that is what makes the representative a true bureaucrat. 
With the division of the Lower House and Upper House (the body that makes the laws) from the cabinet 
(the body that executes the laws), real power fell into the hands of offices in close contact with the banks. So, in the 
end, democratic principles and parliamentary rule were swallowed up by the big bank moguls (Morgan in America, Locheur 
in France, formerly Stinnes in Germany), that is "official democracy": formed by funds.\nline

Thus, the real democracy of today becomes the dictatorship of the bourgeoisie (Cromwellism, Napoleonism and now Fascism) 
hiding behind the press, schools, churches and masked parliaments in the tranquillity of capitalism and real political power, 
like the economy, always lays in the hands of the bourgeoisie.\nline

I have already described Sovietism and parliamentarism in the brochure "Parliament or Soviet" (printed in 1911), so I am 
only outlining the main points here.\nline

In this age of proletarian movement and revolutions, the restless workers put forward all their opposition and 
stand against the rule of the bourgeoisie, just as the bourgeoisie overthrew the feudalists in the 100-year mental and 
physical struggle (1740-1848).\nline

Communist economic order is contrasted with the capitalist order, workers' dictatorship with bourgeois dictatorship, 
Sovietism with Parliamentarianism.\nline

Just as parliament is an invention of the bourgeoisie, the Soviets are an invention of the workers' dictatorship which, 
with the help of the peasantry, overcomes the bourgeoisie. Thus, the soviet is a political instrument in the hands of 
the workers, created before or during the revolution. It was a political state that turned capitalist society towards 
communism by nationalising all the means of production and organising all production and distribution in a communistic manner.\nline

The economic, political and educational institutions set up during the dictatorship were used not only to weaken and destroy 
the bourgeoisie in the political, economic and ideological arena, but also to educate all the forces of society towards communism.\nline

While the workers exercise dictatorship over the bourgeoisie, within the class itself there is already real democracy. 
It has real political power because it controls all the means of production and distribution. In addition, it will have all 
the means of propagating the revolutionary spirit, such as schools, newspapers and much more.\nline

The Soviets sought to destroy the "bureaucracy" inherent in parliamentary arrangements. 
In order to achieve this goal, the following measures were taken.\nline

\begin{enumerate}
    \item The election cycle is shortened.
    \item The relationship between the elector and the electorate shall be brought closer and the legislator and the executive shall be united and one authority shall be established, which shall both make and execute laws.
    \item Representatives may be appointed and dismissed at any time.
    \item The workers should be included in the government as much as possible.
\end{enumerate}

The educated workers, who are supposed to be in charge of the government of the country because the 
bourgeoisie will always make an effort to fight for the defeat of the workers, and this they will certainly do 
by counter-revolution. They will be organised in the communist party.\nline

Under these circumstances, political power will be extended to the organised workers and trade unions and eventually to the entire working class.\nline

Accordingly, each revolutionary class should seize and retain all political power. For when political security in each 
country is established, economic enterprises can be carried out and, with them, real democracy.\nline

Indonesia has never known "democracy". And in the absence of a strong native bourgeoisie, for the time being, 
Indonesia will not be acquainted with democracy. All efforts to achieve it will be unsuccessful, and it can be said 
that all such ideals - dictatorship - bourgeois democracy - are impossible. Only the Indonesian working class can 
hold the dictatorship (if it stays conscious and organized). It controls economic life.\nline

And today, the workers are one of the most organised classes in Indonesia. We shouldn't regret it if we skip the age of "deceptive democracy"!\nline

The political stability of the Republic of Indonesia can be maintained by a worker's dictatorship whose 
revolutionary power is embodied in one "strong" revolutionary party. Over time political power can be extended to every Indonesian worker.\nline

\section{Our "People's" Council!}

The work of a corrupt bureaucracy and great hypocrisy! Indeed, was it only with the Philistines in the past that we find such violence and deceit today?\nline

Where are the people who stand behind the People's Council? And what has the luxurious People's Council done for 
the people? Among the 48 members, 20 are Indonesians and 28 are foreigners representing foreign capital. Under 
these circumstances, all the members' endeavours to win votes are futile.\nline

Even if it were a real council, it could do nothing because all its advice could be thrown into the dustbin by the people in power (the People's Council is not a law-making body, but an advisory body).\nline

The number of Indonesian members is too few and, therefore, they cannot express the will of the people. 
If we remember that the Netherlands, with a population of 7,000,000, has 100 members of the \emph{Tweede Kamer}\footnote[9]{The "Tweede Kamer" is a lower house of representatives in the Netherlands.} 
(the Eeste Kamer is not included), Indonesia, with a population of 55,000,000, should have at least 600 members in parliament.\nline

Among the 20 Indonesian members on the council, not one is truly a representative of the people or elected by the people, 
let alone for the people. Eight were appointed by the Governor-General and most of them were careerists, such as the 
representative from Sumatra, Demang Loetan, and from Java, Dawidjosewojo. Or they were the children of politicians as 
best exemplified by His Excellency Tuan Soetadi. The other members were elected by the gementee (PEB) meetings, the 
evidence of which is quite clear! There is no point in writing about all the rottenness of the Dutch bureaucracy in 
this book. Nor is it worthwhile for us revolutionaries to seriously criticise all the proposals discussed or accepted 
by the council. If we do not want to be deceived by the nice names and sweet promises of this government, we can 
summarise Dutch colonial policy as follows.\nline

\begin{enumerate}
    \item The 55,000,000 Indonesians have no say in politics.
    \item The big capitalists rule through heartless bureaucrats and petty militarists.
    \item The People's Council is "a blood sucking leech" attached to the back of the Indonesian people.
\end{enumerate}

\section{Hopes for the House of Representatives}

Is there any hope that Indonesia will one day have some kind of House of Representatives? 
The answer is a definite: "no". Establishing a House of Representatives during the current 
social and national struggle would mean the death of Dutch imperialism or the " destruction" of its political machine.\nline

This must be known by every Indonesian!\nline

This is not a question of the "maturity" or " ripeness" of the Indonesian nation but, as we have repeatedly pointed 
out in the course of this book, is due to the absence of a modern native bourgeoisie, whose economic interests are 
more or less the same as those of the imperialistic-capitalistic bourgeoisie.\nline

If today the representatives of all or part of the Indonesian people were elected by Indonesians in a free election, 
they would immediately face a class problem. If they don't choose to deceive the electorate, their representatives 
should take up the question of economic, social and political improvements against big capital. These are not small 
improvements carried out slowly by bureaucrats, but radical changes carried out quickly and practically under the 
leadership and supervision of the people's representatives.\nline

For example, thieves such as those in the Rice Company in Selat Jaran and other government companies should not 
be punished with "dishonourable" dismissal as is usually done to petty thieves. Such gentlemen, who are on the 
people's payroll but ruin the people's enterprises, should all be hung "with honour".\nline

If in the future the representatives of the people can make real reconciliation, the people will feel that 
they are materially and morally better off, and the question of the "flag" (colonisation or independence from the Netherlands) 
will be temporarily forgotten. Not because it is unimportant, but because major difficulties can be removed and political 
ideals can largely be achieved.\nline

We will not discuss the form of government that will be established as described above. That is a 
matter of wishful thinking and the structure of the country's government is based on mere "theoretical considerations".\nline

But the question is, will Dutch imperialism be able to make real compromises in the future? If we once again bear in 
mind the contradiction between the Dutch capitalists and the Indonesian workers, the absence of the native bourgeoisie, 
the financial weakness and the shortsightedness of Dutch imperialist politics, we can answer this question with "impossible"!\nline

In conclusion, all the fuss about the change of government in Indonesia that is now being discussed by the Dutch intelligentsia and 
bureaucracy is a waste of time. If the Indonesian people do get a House of Representatives one day, it will not be a "gift from above" 
but a "strong pressure" from below.\nline