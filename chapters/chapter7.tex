\section{In Retrospect}

"Politics" in Indonesia has never been "a common good", the common property of the people. 
The notion of statehood has never gone beyond a small clique of Hindu or half-Hindu colonisers.\nline

As in most feudalistic societies in Indonesia, the government of the country is held by a 
king and his accomplices. The king, having successfully performed the role of the "hero", 
then appointed himself as the reigning king. His son, who could be dumber than a buffalo, would 
later replace his father as the lord of the land. This hereditary rule "vanishes" when a 
new "hero" comes along and overthrows the old one, making himself king of the land.\nline

No constitution determines the coronation or impeachment of a king and his ministers, and specifies 
them carefully. All power and its sphere of influence rests on the king's violence and will, as well 
as the trust and servitude of the masses. Government of the people, for the people, by the people, as 
Lincoln said, has never been known in Indonesia.\nline

Occasionally there is a king who is "fair" on the political stage. However, this is 
an exception, a fluke and an anomaly. There is nothing the people can do in the absence of such 
a king but to revolt. Indonesia has only known the rule of a few and has never known written laws.\nline

The situation in Minangkabau\footnote[7]{Minangkabau is an ethnic group residing in the highlands of West Sumatera.}
is slightly different. Government according to custom is entrusted 
to the representatives of the people, the datuk-datuk, the penghulu. They must rule according to 
certain laws. The supreme power is called \emph{"mufakat"}\footnote[8]{"Mufakat" is a traditional Indonesian term for "to reach a consensus".}, 
which is obtained by legislation in a meeting.\nline

Every meeting must be open to the fullest extent possible and according to established custom. 
Men and women at meetings have the fullest right to speak which shall not in any way be curtailed. 
Whether on local or national matters, "the law" reigns supreme.\nline

However, such a situation only exists in Minangkabau, which is a small, remote area in the Indonesian 
Archipelago. For this reason, the people there were not much influenced by the Hindus and Arabs, in short, in terms of politics.\nline

Despite the Dutch, if they want to treat the Indonesian people with the same respect as their peers, 
for example as in other parts of Indonesia, in drafting and executing laws and in forming and imposing governments, 
"the people must not interfere".\nline

\subsection{Basic Minangkabau Law}

\emph{"Anak kemenakanaja kepada penghulu,\vskip 1mm Penghulu beraja kepada mufakat.\vskip 1mm Mufakat beraja kepada alur dan patut"}.\nline

\emph{(Rougly translates to)}\nline

\emph{"The people follows the leaders,\vskip 1mm Leaders to the consensus,\vskip 1mm Consensus to customs and tradition"}\nline

This was also the case in the kingdoms of Poko-Dato, Srivijaya, Majapahit and Mataram.\nline

Because the people could not intervene in the government of the country, the East India Company was able to 
conquer or compromise with the kings of Indonesia, and gained power little by little, and finally the whole 
of Indonesia fell into its hands.\nline

\subsection{People's Representatives or Soviets}

During the Dutch colonisation, a social discourse was born that gradually 
called for a solution to the question of the organisation of the state but governance was still 
not necessarily parliamentary or Soviet.\nline

Parliamentarianism in Western countries was born by the bourgeoisie when arbitrary rule was 
rampant everywhere and the bourgeoisie, with its advanced commerce and industry, felt that it 
was being hindered in expanding its enterprises by feudal lords: who were hindering it with various 
taxes and fees, while it was denied political rights. It was under such circumstances that 
the Magna Charta, Cromwellism and the French Revolution were born. Voltaire, the great leader of 
the bourgeoisie, then attacked the Catholic religion and its priests and preached "atheism" (denial of God).\nline

Rousseau opposed autocracy with democracy and to oppose hereditary rule, he preached the "social contract", which 
is a government that contracts with the people. According to Rousseau's doctrine, a king should only rule as 
long as he acts in accordance with the contract; the people should oppose him if the contract is broken.\nline

Because the French bourgeoisie felt that it lacked the strength to resist the rule of the king, the nobility and the 
clergy, it united with the revolutionary masses, the workers and peasants. However, these masses were not allowed to 
take power. They were to be used as cannon fodder in the bourgeois revolution, while power was to be held by the 
bourgeoisie. With the slogan of "Liberte, Egalite and Fraternite", now democracy, liberalism and parliamentarism, 
they were able to overthrow the feudalistic government.\nline

After gaining political power, "bourgeois democracy" shows itself. Although in parliamentary states, 
such as England, France and America, every citizen is given the right to vote, the workers and the poor 
there (the largest number of people) are always unable to support their candidates in parliamentary 
elections because they are trapped under the influence of the bourgeois thought developed in schools, 
churches, newspapers, and above all, because they lack the means of propaganda (meeting rooms, newspapers and brochures, 
all of which are expensive).\nline



\section{Our "People's" Council!}

\section{Expectations for the House of Representatives}