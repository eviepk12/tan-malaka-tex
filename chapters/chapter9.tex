With a diverse range of opinions, in different circumstances and by different sections of the people, 
our political goal has been declared to be national independence. People all across Indonesia are 
unanimously united in this ultimate goal. It is only on the path to be taken and the means to be used that opinions differ.\nline

The rapid change from feudalistic to capitalistic state structure, which was not in accordance with the will of nature, 
caused the Indonesian people to change their way of thinking rapidly. However, this change in thinking usually lags behind 
economic change. Generally, our nation appears superficially modern in accordance with the capitalist era, but its way of 
thinking is still old-fashioned, still living in the past, such as still adhering to the Mahabharata, Islam, and various kinds 
of superstitions and beliefs in ghosts, jinn, supernatural powers, sacred stones and others. They still continue to be childlike 
and fantastical.\nline

The defeat in economic competition with the stronger Western capital has led to the emergence of incorrect and anarchistic 
thinking that does not see things in their true light. This is especially the case among the newly defeated and suppressed 
inhabitants of the small villages and some of the young industrial and agricultural workers who have been dispossessed of their property.\nline

Just as there are different levels of industrialisation, so too there are differences in the minds of people in different parts of 
Indonesia. Let us just point out the difference in the progress of thought between the people of Java and our brothers in Halmahera, 
or between those in Surabaya and Semarang who are more conscious and the unindustrialised villagers. Where capitalism grows 
and takes root, rationalism and sound thinking come to life and superstition gradually disappears. Thus, the psychology 
and ideology of the soul and mind of the Indonesian people are in line with the ever-changing development of capitalism. 
The old disappears and the new becomes more developed.\nline

It is very difficult to bring all the different ideas that are undergoing transformation to the same 
constructive and unchanging ideal. It is therefore a very difficult task for the revolutionaries to bring the entire 
Indonesian people in line with and in harmony with marxist actions. It is easy to slip into profiteering, anarchy and 
believing in talismans.\nline

Until now, no party has been able to draw a line that fits the existing conditions in Indonesia and lead our people along it. 
Several parties have successively lost their way on this road that does not lead to the desired goals.\nline

Believing in the peaceful parliamentary way, i.e. paving the way for Indonesian independence by fighting for 
seats in the People's Council and begging to be granted political power, we call it a misleading "experiment in profit". 
This experiment can only be considered theoretically and pragmatically in a colony with a native bourgeoisie. 
Honest co-operation with the Dutch colonialists outside or inside the People's Council is a betrayal of the Indonesian people.\nline

This does not mean that we will forever turn our backs on the People's Council. On the contrary, if tomorrow or the day after 
tomorrow we get the opportunity through direct elections to occupy the People's Council, it is our duty to enter it. It 
would be wrong and cowardly of us not to do so. However, not for a minute do we intend to work together in the People's 
Council with the sugar robbers, oil thieves and sap robbers, we are forced to enter it, oppose them, take courageous opposition 
action, and shatter their facade. We use the People's Council as a "People's Court" and obstruct the government's actions from within. 
By doing so, we can adequately educate the people who are not allowed to write and speak politics outside the People's Council.\nline

We consider it foolish to use a method that is in direct contradiction to the one mentioned above, because it is more 
detrimental to the cause of independence than most people think. As long as one believes that independence will be achieved 
by means of "putch" or anarchism, it is only a fever dream. And the development of this belief among the people is a 
misleading act, intentionally or unintentionally.\nline

"Putch" is the action of a small group that moves secretly and has no contact with the masses. It can only plan according to 
its own whims and skills without regard for the feelings and capabilities of the masses. It suddenly came out of its cave without 
first considering whether the time for mass action is ripe or not. It assumes that all of its daydreams about the masses are 
completely true. It forgets or refuses to understand that the masses can only be successively drawn into violent political action 
(in the modern sense!) and in times of misery and blind reaction. The "putch" forgets that at this moment of revolution when 
mass action turns into armed revolt cannot be determined months in advance, as a "putch" can. "Revolutions arise naturally as a 
result of various circumstances". If the "putchers", at a time appointed by themselves, came out suddenly (like Herr Kapp the famous "putch"), 
the masses would not come to their aid. Not because the masses are stupid or unobservant, but because "the masses only fight" for 
the needs that are immediate and in accordance with their economic interests.\nline

Not a single political victory, until now, has been won by the masses (not by a military gang!) if not by economic or political action! 
Often at first people go the legitimate way. However, because the putch-masters went off the legitimate path, i.e. suddenly resorted 
to the violence of the weapons of governmental aggression, 99 times out of 100, they were abandoned by the masses because they had 
isolated themselves from the masses from the beginning. Likewise, 99 times out of 100, the putch "plot" is discovered by the enemy. 
Putch plots are forever leaked because half the members are impatient and they talk or because of the betrayal of a frightened member. 
Or their movements can be spotted by spies roaming around.\nline

To putch in a country like Indonesia (especially in Java), where capital is neatly centralised and protected by a modern Western military 
and spies - in contrast, the people still believe in the supernatural, superstitions and fairy tales - is to "play with fire": one's own 
hands will be burned. The anarchists who used to say that this solid Western power could be brought down with a few "popped" eggs were no 
more cunning than the man who smashed a rock with his head.\nline

Only "one mass action", that is, a planned mass action will win, in an industrialised country like Indonesia!\nline

Mass-Action does not recognise the empty fantasy of a putch or an anarchist or the courageous act of a hero. Mass-action comes 
from the masses to fulfil their economic and political will. It is caused by great misery (economic and political crisis) and is 
ready, whenever possible, to turn violent. A party based on organised mass action is definitely capable of bringing about mass 
action that breaks up a calm and safe harbour.\nline

Part of mass action manifests itself in "strikes or boycotts". If workers by the millions were to lay down their jobs for a specific purpose 
(demanding economic and political benefits) the losses and economic turmoil caused by their actions would undoubtedly weaken the 
oppressive colonisers.\nline

According to our strength and victory at that time, we were able to gain political and economic rights. In India the boycott turned out to 
be a double-edged knife. On the one hand it severely injured British importers, on the other hand it promoted native trade. In Indonesia, 
the absence of substantial native capital made the boycott of foreign trade even more difficult.\nline

Not only is this great power due to the endeavour to accumulate the necessary capital, but also to continue the boycott. 
We can easily surmise that the great and violent Indonesian national boycott is hated and resented by the savage Dutch imperialist 
just as much as he hates general strikes.\nline

However, boycotting in Indonesia is not an impossible task. On the island of Java and beyond, there is not only a lot of small native capital, 
which, if pooled into national co-operatives, could create a huge capital. But such an endeavour would require too much of the entire 
Indonesian population to be aware and active.\nline

Tax boycott, which was considered to be an action in India, was never carried out because of the bourgeoisie's fear of revolutionary 
repercussions. In Indonesia tax boycott is a very powerful weapon of political economy.\nline

However, such actions amount to "breaking the law" and only occur in revolutionary conditions under the leadership of a strong revolutionary party.\nline

The political part of mass action manifests itself by demonstrations and in India by collective labour resistance with political 
and economic intent, demanding home rule from British imperialism. The political part takes the form of the following acts of abandonment:\nline

\begin{enumerate}
    \item governmental bodies
    \item government courts
    \item government schools; and
    \item the police and the army.
\end{enumerate}

The fourth measure, for fear of rebellion, was never carried out. The first through to the third were not carried out long enough and 
did not produce enough results. Can they be carried out longer and more successfully in Indonesia than in India? We will answer this 
question later in a special discussion.\nline

Political demonstrations are organised by masses marching along streets and in meeting halls, with the intention of protesting and 
reinforcing political and economic demands and showing the enemy the extent of our power. If "slogans and demands" are truly shouted by 
the masses, a political demonstration can become a great wave, which becomes stronger and stronger the longer it goes on, breaking down 
the economic and political strongholds of the ruling class.\nline

In an industrialised country like Indonesia, "mass-action", i.e. boycotts, strikes and demonstrations, can be used more fully as a 
sharper weapon (in India this is not the case because the native capitalists fear general strikes and the political power of the workers, 
a fear not so different from that of the British bourgeoisie!)\nline

If a revolutionary party succeeds in mobilising millions of workers to leave their jobs and non-workers to refuse to cooperate and the 
whole people to demonstrate for economic and political rights without throwing a single pebble at government officials, the moral-political 
consequences of this action will be enormous. It will bring more benefits in the political and economic struggle than a hundred Jambi 
Rebellions or riots, bizarre killings and done by dashing putchists. We must not forget that the action we are about to take is now 
prohibited by law, but there is no reason for us to abandon this one and only path.\nline

In addition, it is a big question whether the government can maintain the prohibition, at least if it is not quickly discouraged by minor 
defeats as in the past. Genuine human rights such as strikes (refusing to sell one's own labour), boycotts (refusing to work together, 
buy or sell goods) and the right to demonstrate (declare one's ideals) will disappear forever from the Indonesian nation if behind every 
Indonesian stands an armed imperialist soldier.\nline

The advantage of mass action over putch, is that with mass action our struggle can be safeguarded, whereas with putch, 
we show envy to the enemy. In mass action, the leader can go as far as is reasonable at the moment. He can forever determine how far he 
can go in making political and economic demands without incurring great losses (sacrifices are necessary in every mass action). And it 
does not lose contact with the masses. Likewise, the relationship between the masses themselves is not broken. With a surprise attack, 
i.e. a deliberate putch against the enemy, they are from the outset vulnerable to attack by the enemy. The leader of the mass action, 
holding the "map of struggle" in his hand, can play the enemy by taking small steps forward and then hitting them once.\nline

Mass action needs a leader who is revolutionary, intelligent, agile, patient and quick to calculate upcoming events, 
politically aware. He must also work with the national forces that already exist and not hope for a force that is merely a daydream. 
Furthermore, he must know the character of the masses he leads (knowing when and how the people will react to political and economic events).\nline

They must also be able to use slogans that energise the people and turn "mass will" into "mass action". In addition, he must know the 
political and economic positions well and be able to use them without hesitation. Since the ruling class (the government) 
has a well-equipped army and is always on standby, the skill and agility of the leader of a modern movement of mass action must have a 
practical knowledge of the politics and economy of the country as well as the psychology of the people and then be able to calculate 
the political events that will occur. Moreover, the leader must be able to use "time" quickly and correctly, as well as to exploit all 
contradictions within capitalistic society (as well as within the army) to his advantage.\nline

So, if " idiots" (as in feudal times) can putch, a modern leader of a mass movement must be a man of intelligence and wisdom.\nline

\section{Parties and its Characteristics}

What is a party? If we want to gather and concentrate the revolutionary forces in Indonesia by means of planned mass action to pave the 
way for national independence, we must have a revolutionary party. However, until now Indonesia has not had a revolutionary party, 
there have only been associations of people with "different" political views and actions. A revolutionary party is an association of 
people who share the same views and actions in the revolution. And the best revolutionary action is that of each member together, 
with each other, concentrated.\nline

In order to eliminate any ill-feeling among party members, each person should be given the right to speak out, express and defend 
their beliefs as much as possible. And a party decision should be regarded as the result of deliberation and careful joint consideration 
of all members. Every deliberation should be conducted in a truly democratic manner. Every hint of bureaucracy and aristocracy must be 
rooted out. However, bureaucracy and autocratism in the party cannot be eliminated by "cursing" or by pounding the table, but by getting 
into the habit of free exchange of ideas and co-operation of all members. Every party decision must be taken by majority vote. If a 
decision has been accepted by the majority, the smallest vote, even if it is against its convictions, must "submit" to the decision 
and honestly carry it out. Otherwise, a party will never achieve revolutionary vigour. A decision that is "half correct" but happily 
carried out by the whole ranks is better than a decision that is "extremely good" but betrayed by half the members.\nline

The party must have "iron laws". Only then will it be able to centralise party action. The party must have revolutionary tools to check and 
correct the actions of all members. It is not enough for a member to "admit that they agree" with a party decision or rule. They must 
prove by their actions that they are carrying out the decision correctly and that they are loyal to the party. These actions usually 
include, for example, seeking comradeship in party newspapers, courses, trade unions and working on party administration and organisation. 
If they do not fulfil these or it is "proven" that they are not loyal to the party, they should be disciplined. It is better for them 
to leave the party than for them to damage the party or set a bad example as a lazy revolutionary to other members.\nline

\section{Our National Programme}

The political, economic and social revolutionary aims of a party for a particular country and the path it will pursue together are outlined 
in a revolutionary "national programme". The programme is the guidepost for the party and must be recognised, understood, defended and 
developed by every member. I have described our national programme and its general characteristics quite clearly in the brochures 
\emph{Naar de republik Indonesia} and \emph{Semangat Muda} (issued in April 1925 and January 1926 respectively). I shall not elaborate on the matter here, 
and will leave it to the reader to peruse those booklets. However, for the convenience of the reader, I have attached the national programme 
(not annotated) to the back of this book.

\section{Party Duties and Organisation}

The party pursues the cause and is the vanguard \emph{(avantgarde)} of the movement at all levels of the revolution. It is far-sighted and always 
fights at the very front and, as such, it is the "head and heart" of the revolutionary masses.\nline

In the French "bourgeois revolution" (1789), the \emph{avantgarde} consisted of the revolutionary bourgeoisie and the bourgeois educated workers.\nline

They are the ones who lead and direct the revolution, while the weak industrial workers are used as "slave labour", as mere pawns. 
This may also be the case in colonised countries where the native bourgeoisie is strong but is denied political power by the coloniser, 
so they are forced to become revolutionaries. In Egypt and India, the driver of the independence movement has hitherto been the bourgeois 
intelligentsia.\nline

Those who fought in the colonial countries were mainly revolutionary workers and peasants. In Indonesia the native bourgeoisie could not take the lead, morally and materially.\nline

Because the social and economic conditions are too weak, the workers must establish their own ideals and organise their own army. 
If the bourgeoisie, big or small, in Indonesia want to enter the masses, they must not fight with national capital and parliamentarism but 
they must stand on the principles of labour, nationalisation and workers' and peasants' government. They must become educated workers and 
fight with the workers for workers' ideals and with logic.\nline

If the bourgeois intelligentsia want to be recognised by the masses as comrades, they must do more than their class compatriots in Egypt, India and China.
As a class, of course, they cannot do so because they are hindered by their own lineage, education and environment.\nline

The working class in Indonesia cannot expect all the educated workers of our bourgeoisie to throw themselves into the struggling masses 
tomorrow or the day after tomorrow. But some of them (not as a class) "may" enter the new ranks as voluntary soldiers. The revolutionary 
bourgeois intelligentsia, if it is uncritically incorporated into a revolutionary workers' party, is "bourgeoisising" our workers. 
In Indonesia, especially, it means "castrating", depriving the workers of their revolutionary feelings and advanced ideals. 
It will not be possible to get energy out of such a labouring class. Such a party, "neither fish nor meat", is not proletarian revolutionary bourgeois.\nline

Even if the Indonesian bourgeoisie were stronger and more revolutionary than it is today, it would not be willing and able to go further than "political independence", i.e. seizing political power from Dutch imperialism.\nline

Radical solutions to economic and political problems (assuming there is a strong, revolutionary Indonesian bourgeoisie) can only be 
implemented at the expense of native capital itself. To such a solution, the bourgeoisie in question would undoubtedly not consent. 
In every colonised country, the revolutionary (against imperialism) native bourgeoisie immediately turns into a reactionary class the 
moment imperialism is overthrown. The ultimate aim of each revolutionary native bourgeoisie is purely "political". In India, China, Egypt 
and the Philippines this has already been proved. So too have the multitudes of the Indonesian petty bourgeoisie. Hidden in their political 
struggle against Dutch imperialism are aspirations for greater wealth and power. They want to become landlords, wealthy merchants, bankers 
and also want to become governors, ministers and so on. In short, they wanted to become big bourgeoisie, as in other countries. 
The ratio between capital and labour, between capitalists and workers and the political system, all three of which they want to 
remain capitalistic. By overthrowing Dutch imperialism, the Indonesian petty bourgeoisie want to be able to exercise all political and 
economic power over the workers.\nline

The workers' aims go beyond the boundaries of "anti-imperialism". They intend, whether explicitly or vaguely, to overthrow the capitalists 
altogether. The Indonesian workers want a radical economic, social, political and ideological solution, now or later. If and when the Dutch 
imperialists are opposed and crushed to the roots, although this is not possible in the sense of a purely national victory, the workers will 
and must strengthen their ranks against the bourgeoisie.\nline

So, the small Indonesian bourgeoisie, let alone the big one, is only anti-imperialist, while the workers are against both: imperialism and capitalism.\nline

So, the Indonesian workers, compared to the revolutionary bourgeoisie, have a much longer way to go before they reach true independence. 
So, they should be more vigorous and radical in their struggle and they already are, just like in other countries.\nline

The "question of organisation" is closely related to the social, economic and political ideals and the revolutionary position of 
the revolutionary classes. According to the ideals and " looks" of all the revolutionary classes, we can divide our national army into: 

\begin{enumerate}
    \item \textbf{Vanguard}, which consists of conscious industrial workers and educated workers;
    \item \textbf{Reserve}, which consists of less conscious workers and non-revolutionary workers who during the revolution fought under the leadership of and stood by the side of the vanguard.
\end{enumerate}

The agitation must be based on the actual lives of the masses. It is not enough to shout independence. We must show independence in its true
colours. We must explain all the daily sufferings of the people such as wages, taxes, hard labour, squalid housing, humiliating and cruel 
treatment of the people. A skilful agitator must at all times be ready to solve all questions relating to "Pak Kromo" material life in a 
correct and revolutionary manner. He must also always be willing to attract and lead "Pak Kromo" to political and economic action that 
improves their material needs. We cannot expect the masses to enter the struggle on the basis of ideals alone!\nline

The masses (in the East or in the West) only struggles because material needs are paramount. With economic struggles,
such as strikes or boycotts and supported by political demonstrations, we will be brought to the ultimate goal!\nline

All agitation must be suited to the circumstances of each region. Information to an industrial labourer should not be confused with a peasant 
because both have different material needs. Nor should a peasant in Java be confused with a peasant in Sumatra because both have different 
problems of land and economy.\nline

If the agitation is real and recognises all the needs of the depressed people in every region of Indonesia, 
if our programme of demands and slogans are "really" understood and felt by all strata of the population, if the party leaders are keen, 
agile and intelligent in making use of all the existing conflicts in Indonesian society, 
the party can undoubtedly gain the necessary connection "with" - the desired influence "over" and finally the necessary trust "from" - the masses.\nline

This article is already longer than we originally intended, especially when we add to it the discussion of the "techniques" of mass action. 
Even this should be left to practical discussions because we do not want to "expose" ourselves to the enemy by revealing the secrets of our 
techniques of struggle. However, here we must warn that the question of " weaponry" - although it is very important and attracts 
the attention of revolutionaries very strongly! - is not a question of life and death for us. It is subordinated to the question 
of revolutionary politics and organisation. In other words, the joyous masses under the leadership of a revolutionary party with a 
steel discipline, fighting with their hands and the sound of revolutionary chants, will crush the imperialist army to the core.\nline

To conclude this chapter, we may add that for revolutionary victory, the following two factors are necessary.\nline

\begin{enumerate}
    \item \textbf{"Objective" factor}, i.e. a level of the productive forces and the destitution of the masses. This level, especially in Java and in some parts of Sumatra, is considered sufficient in our view.
    \item \textbf{"Subjective" factor}, i.e. the willingness of the Indonesian people which must be manifested in a "perfect" (well-organised and mature) revolutionary party and good revolutionary conditions.
\end{enumerate}

To achieve this, the party must have discipline; the dissatisfied masses must be subordinated to the leader. 
Then the domestic and foreign enemies must be divided. See next \emph{Naar de republik Indonesia} article "strategic blows".\nline

Even if a revolutionary party cannot be obtained by academic discussions within the party or there is no opportunity for our miserable 
and humiliated nation, we can always push the party into great economic and political struggles or create the desired "discipline" 
that gives the masses an irresistible influence and the necessary confidence and, above all, intelligence in the struggle. These are 
the conditions that will lead us to victory.\nline

The weak middle class and bourgeoisie would only fight if they were forced to.\nline

It would be too long to discuss here at length the question of one or two parties. What we mean by that is whether the workers and 
the petty bourgeoisie should be brought together in "one" national organisation with "one" central leader or split into "two" 
organisations with two leaders but working together (At the moment the workers - because a definite system has not yet been adopted - 
can be said to be not yet organised in the Indonesian Communist Party (P.K.I.) and the non-workers in the people's unions. 
Both have one big committee).\nline

No matter what form the organisation takes in a colony like Indonesia, the workers are the most active and radical. 
The organisation must not hinder that activity. On the contrary, it must know how to use it and be capable of bringing it to life. 
The organisation should be the fusion and concentration of all workers' activity.\nline

Efforts should be made to get as many workers as possible in the party and at the helm. 
Our revolutionary party will flourish to the fullest and healthiest extent when the seeds of the party are sown in every industrial centre.\nline

Thus it came to pass that the P.K.I. was confined to the cities, the economic centres, transport; and the People's Union (S.R.) 
had to become a non-workers' party. In addition to the cities, it should also be established in the villages. In this way, 
revolutionary fire should be infused into the P.K.I. and the S.R., and the half-awakened and unawakened workers should not 
remain outside the organisations at all. They must be brought into the economic struggle, which at any moment turns into a struggle; 
they must be brought together in the trade unions as a reserve ranks standing directly under the leadership of the P.K.I.\nline

The half-awakened and totally unawakened non-labourers in politics and economics are also being compelled to be gathered into 
the people's cooperatives, which are also auxiliaries standing directly under the leadership of the P.K.I. and the S.R.\nline

Thus, the P.K.I. must have several organisations of trade unions, cooperatives and people's unions, each of which, in mass action, 
is directly under the leadership of the P.K.I. These organisations - whose spirit is influenced by the party and trade union newspapers - 
are the soldiers of the national revolution in the struggle against imperialism and Western capitalism.\nline

If a revolutionary party really wants to be the leader of the masses in Indonesia, it must first lead itself well. 
Party organisation is the sum of all party structures. In other words, it is the " lifeblood" of the party, being the "most important", 
such as the organisation, training, education of its leaders and members. In addition, the party must be in close contact with the masses, 
especially at important times, with all groups of people from all over the Indonesian Archipelago. Without such a relationship, 
there can be no revolutionary leadership.\nline

A member of the P.K.I. should as far as possible be an educated worker or a worker (not bourgeois). 
He must know and be able to explain communism in theory and practice, national and international tactics. 
Above all, he must be more and more equipped to do revolutionary work, that is, the work of organising and making comrades. 
A member of the S.R. is usually not a labourer, peasant, merchant or student. He does not need to do as much revolutionary work as a 
P.K.I. member, it is enough if their ideology is anti-imperialist and they want national independence. Under the one-party system, 
workers and non-workers are brought together in a revolutionary organisation. Within the party, the more "conscious" and educated workers 
constitute the left wing. This left wing is the motor of the Indonesian movement.\nline