\section{Ways of Exploitation and Opressions}

"God created the world in His own image."\nline

The foreigners who colonised Asia for 300 years were to fulfil 
their own needs and they ruled the conquered lands in various ways. 
Economically, from the past until now can be divided as follows.\nline

\begin{enumerate}
    \item \textbf{Open Robbery}, formerly carried out by the Portuguese and Spanish.
    \item \textbf{Monopoly}, which in practice is equivalent to robbery, is still being carried out by the Dutch in Indonesia until now (c. 1926, ed.).
    \item \textbf{Half Monopoly}, began to be carried out by the British in India.
    \item \textbf{Free Competition}, started by America in the Philippines.
\end{enumerate}

Other imperialist methods can almost be compared to the ones mentioned above.\nline

The ways of oppression in politics are as follows.

\begin{enumerate}
    \item \textbf{Barbaric imperialism}, which is to destroy all indigenous political power and run an arbitrary government, for example is Spain in the Philippines.
    \item \textbf{Autocratic imperialism}, which is almost indistinguishable from the aforementioned article “a” like the Netherlands.
    \item \textbf{Semi-liberal imperialism}, i.e. imperialism that gives very limited power to the ruling indigenous people (kings or heads of state for generations such as Britain in India).
    \item \textbf{Liberal imperialism}, that is, imperialism that gives complete independence to the great landlords as well as to the rising indigenous bourgeoisie, for example is American imperialism in the Philippines.
\end{enumerate}

\section{Causes of Difference}

The differences in the methods of extortion and oppression of the colonised are 
not due to differences in human nature in the imperialist countries. 
But because of the position of the capital of each country when they 
arrived in Asia, and also the way the capital was run.\nline

By the time the Spanish and Portuguese arrived in Asia around the 
1500s, they had not been completely free from feudalism. The Portuguese 
and Spanish were countries of agriculture, handicrafts, nobility and 
religion (so there was no victorian style industry yet).\nline

There were no industrial goods that could be sold in the colony's 
markets. They came to the colonies to rob the produce there and then sold 
it on the European market at high prices. Because they strongly embraced 
the Catholic religion which had just expelled Islam from Spain, the Indonesians 
who embraced the animist religion in the Philippines were forced to become 
Christians. Who does not like to follow the force beheaded with a sword.\nline

By the time the Dutch followed Spain and the Portuguese to Indonesia around the 
1600s, much of Dutch feudalism had been pushed by the bourgeoisie. They had broken 
free from the oppression of feudalism as well as Catholicism and took the path of 
free trade, liberalism and Protestantism. The Netherlands was in the age of young capitalism.\nline

Britain, which in 1750s was able to stand still in India, had in fact been 
immersed in the bourgeois revolution under Cromwell for 100 years.\nline

After that, British capitalism advanced very rapidly, accompanied by the 
notions of free trade, liberalism, constitutionalism and the belief in independence.\nline

America arrived in the Philippines in 1898 after two bourgeois revolutions (1775 and 1860). 
He firmly held Monroe's understanding of democracy and open-door politics.\nline

\section{Consequences of the Various Ways of Extortion and Oppression}

As a result of the robbery, the Portuguese and Spanish were eventually expelled from their colonies.\nline

Although the revolutionary spirit in Indonesia is mature and burning, our supplies are not enough, so Dutch imperialism is still standing.\nline

By granting large concessions, if forced, as well as a politics of compromise to a group of Indians, British imperialism still stood there.\nline

Under the guise of nurturing, helping and loving human beings as well as giving economic 
autonomy, great economic and political freedom to the indigenous people in the Philippine, 
American imperialism still created chaos there.\nline

\subsection{India}

Although Warren Hasting and Lord Clive killed and robbed, 
their actions should not be equated with those of Daendels, van den 
Bosch and others, because the British colonial system in terms of 
"material and history" was much softer than the Dutch system (of course we do not want imperialism of any kind). 
The lust to kill and plunder from British imperialism cannot destroy the will of the Indian people.\nline

This showed itself especially with Indian goods which had not been robbed by the British. 
After experiencing several political and economic struggles, the Indian nation was able to 
establish a large national industry, agriculture and trade. In addition, British imperialism 
organised schools from the lowest level to high schools (more than five universities) 
and for a long time has developed a system of government to "dominion" or beyond. India 
has had Tilak, Mahatma Gandhi, Das, Tagore, Dr. C. Bose and Dr. Naye who are famous all 
over the world. All of these educated people were born in recognition of British imperialism.\nline

Because England in its own country had materials for industry (coal and iron), it in itself 
became a world workshop. Because it did not have cotton in the beginning, India was made a 
cotton plantation. In addition, as an industrial country with enormous incomes, the UK needs 
markets. Because of this, the British lands (the only industrial country) were forced to work 
together with India, albeit at the beginning indirectly. Do not firms and sealanes, whether 
import or export in such a large trade between Britain and India, need the middle Indian 
merchants as intermediaries? And again, can't "bayonets" always force a nation to buy things? 
Inevitably he must raise his standard of living, if he wants to obtain a permanent purchase. 
This is what forced British imperialism to provide Western education to a group of Indians. 
The first High School in Bengal which is now 100 years old, which at first could only be entered 
by the children of the rich and the aristocracy, was later allowed also for the children of the common people.\nline

In such a short time, the high schools also produced so many educated people that the British 
bureaucracy could not accept them at all. There arose a class that was educated in the West and 
who felt unhappy, namely the educated or professional workers. From this class were born some of 
the leaders of the independence movement who are known as extremists, namely the left. Thus, 
British imperialism gave birth to its enemies and dug its own grave.\nline

Under the leadership of the famous Tilak, there was a boycott of 1900-1905. 
The point is to keep national industry and trade alive, namely by boycotting 
British factory goods imported into India (cotton is grown in India, then sent 
to England, at double the price it is also sold to Indian buyers).\nline

By using items that have not been robbed "as weapons", the educated gain victory. 
Large landlords and merchants provided assistance in the form of capital, enthusiasm 
and tools to meet the programs of the extremists. Despite being full of political, 
economic, financial obstacles and extraordinary tactics, Tilak and his compatriots 
were able to win. Various industries, including the weaving industry-the national 
industry of today-are the most important handicrafts of Tilak and his comrades. 
Even the industry already has an international field. Much of the victory also depended on the help of Indian workers and peasants.\nline

Standing on Tilak's victory, Mr. Gandhi won the victory in the non-cooperation movement 
or the boycott movement. Almost all weaving factories in Bombay (approximately 200 in number) 
are now owned and managed by Indian brains and personnel. British cotton was hit in fierce 
competition, not only in India but also in Africa, Malay, China and eventually also in Europe.\nline

India's current trade laws protect Indian cotton. Not a few plantation firms and banks 
are not working with Indian capital and are led by the Indian nation. Industries such 
as charcoal and iron; and the modern metal industry is now held by the Indians. During 
the world war England bought a train wagon from "Tata Coy", now (since about 2 years) 
he made an agreement to also buy train engines. In short, without the violence of British 
imperialism, India's national capital was established — which resulted in a relentless 
struggle, which at times led to bloodshed. India is now in the age of modern big industry. 
England is no longer the centre of the workshop in the world though in its own kingdom; and 
India is no longer a cotton field for Britain.\nline

After Britain was defeated in the economic game, it was forced to acknowledge 
India's victory in politics. There now stands a national industry whose material 
interests are in some respects equal to those of the colonialists. It remained for 
Britain to grant political concessions to the representatives of the great landlords and the modern bourgeoisie.\nline

Indeed, this is the meaning of the reform work of the state government that has been 
done for many years-the Montagu-Chelmsford reforms. Large districts with a population 
of 50,000,000 such as Bengal and the Central District after reforming (hervorming) 
with the intermediaries of district councils, almost fell into the hands of the Indian 
nation entirely. The election of the highest council (Duma of the Indian nation), 
influenced by the Swaray, the military, the academies, and the judiciary, in recent 
years was provided - occupied by the eloquent and loyal Indian sons.\nline

However, there is not yet a single people's representative (parliament) 
and cabinet in charge. Although the reform of the Indian government is much 
more perfect than the Dutch -style House of Representatives, it has not yet 
reached the Dominion of Canada, the Philippine or Egyptian constitution. But a 
number of leaders and extremists could be drawn to the reform. As a result, the 
revolutionary movement was temporarily "stranded" until British imperialism had a chance to catch its breath.\nline

\subsection{Philippines}

The situation in the Philippines is slightly different from that in India. 
The Americans came, in 1898, a time when the Filipinos had a "three-quarters success” 
in overthrowing the Spanish. At first America happened to be an ally, but once her stand 
was solid she stayed in the country. The 33 year Philippine-American War (1898-1901) 
failed to drive away the thieves. Before the arrival of America, the Philippines has 
been able to show some great nationalists such as Dr. Rizal (who was shot by the Spaniards from behind); 
the organiser Bonifacio, Mabini the diplomat and great generals Luna and Aquinaldo.\nline

Therefore, it is necessary to use a very cunning deception to deceive the eyes of a 
nation that is as strong and clever, such as the Filipino people.\nline

Due to the greatness and wealth of America and by one of the influential anti-imperialist 
ideologies among the American people, the imperialists immediately worked for reform. 
Domestic politics, mediated by the "Senate" and the "House of Representatives'', can now 
be said to be in the hands of the indigenous people. All representatives of the two councils-exempt 
from some Islamic regions-were elected with full suffrage and were all Filipinos. Most of the 
governors of the regions are also Filipinos. Only a few department heads are American. In a 
constitution, America must promise to grant the widest possible "independence" "to the Filipino 
people once they have demonstrated the competence to establish a permanent government".\nline

Primary schools were observed with great care and emphasis on agriculture.\nline

The companies that are the backbone of the Philippine economy are now held entirely 
by the indigenous hands. Some factories, trading houses and shipping lines are owned 
or operated by Filipinos. Four Universities and several high schools each year graduate 
Filipino sons and daughters in large numbers to defend the nation of 12,000,000 souls from American deception and fraud.\nline

Very few people are illiterate. Almost all children go to school. To the far corners, 
apart from their own language, the young men understood English.\nline

Even though the colleges there did not please an educated Dutchman like Dr. Nieuwenshuis 
who of course will forever lick the boots of his own government, while insulting the actions 
of others, but because of the height of Filipino intellect, the great and wealthy Americans cannot do as they please.\nline

Because America in 1925 had to pay the price of rubber f 540,000,000 more than in 1924 to 
the United Kingdom, the Americans thought of opening a plantation in the Southern Philippines whose land was good for rubber.\nline

But Filipino leaders are working hard to avoid being targeted by the American "rubber wolf." 
Before they went any further to acquire large tracts of land for rubber plantations, in 
concessions-thanks to the efforts of Philippine leaders, members of the Senate and the House 
with its long-standing land law stipulated that "no more than 2500 acres (one acre 4840 yards square) 
which can be rented to foreigners. Not long after the rubber wolf, through Firestone came to ask for a 
concession for the rubber plantation. They were greeted with the words that Philippine land law "does not grant permission".\nline

Philippine leaders argued that if America invested its capital in the Philippines, in addition 
to the people would soon be miserable (as in Java) also America would have a reason to obstruct 
Philippine independence. American imperialism, which is no less ingenious than Anglo-Saxon imperialism, 
can one day say that a shock may arise because of America's untimely departure? American interests are at stake in the Philippines.\nline

This is why the Philippine leaders hastily removed the land law from the law book and disclosed it 
to the entire people. It was like the village in the arrival of a tiger.\nline

A nation that has already built national consciousness like the Philippines, moreover given 
insight by indigenous newspapers (due to the high school cursed by the educated Dr. Nieuwenshuis!), 
Can see and carry out the truth from its leaders. Accompanied by the entire people, the Philippine 
leaders can at any time shoot the rubber wolf of American imperialism with the arrow of the land law they made.\nline

No one criticised the non-national school system other than the Philippine leaders themselves. 
In addition, there are difficulties in taking on the role of trade from foreign nations. But they 
all agreed that a healthy education system and the best economic changes could only be done 
perfectly after the nation's independence was achieved. And in which corner of the world is it 
viewed differently? The existence of a Governor-General who has the right to prevent (recht van veto) 
is an obstacle to economic reform that is solely for the benefit of the Filipino people. That is why, 
our brothers and sisters in the north there are still fighting solely for independence to the fullest extent possible.\nline

The huge concessions, which were forcibly granted by the United States 25 years ago, 
could not cool the hearts of the Filipinos to take away their birthright and independence.\nline

If the Filipinos had not joined America (one of the strongest and richest countries in the world), 
but "the famous pirates on the shores of the North Sea (Netherlands)”, it would have been a long time 
since the great Filipino people had driven them into hell.\nline

Britain controls more than two-thirds of rubber and America consumes 72\% of the world's produce. 
Due to the "Stevenson Rubber Restriction's policy" still in effect, the garden owners and monopolists, 
the British alone controlled the world's rubber - \emph{American Chamber of Commerce report chamber version 
published in the Manila Tribune, July 26, 1925.}\nline

\subsection{Indonesia}

The situation of India and the Philippines which I have presented above, 
I intend to add to our knowledge of imperialism.\nline

About Indonesia, now and in the future, at length. After considering all 
that is described above, it is certainly not for the reader to define the 
robberies, arson, and murder committed by the Dutch. Therefore, we will not 
linger to describe “hongi-hongi” (pepper in Ambon), a coffee plantation now 
called an independent grower. All of them has been famous and cursed by every brainy human being.\nline

Far be it from us to say that all these incidents were merely the work of 
individual Dutch people. We ourselves have known enough of the character and 
habits of the Dutch. But the manners and songs of Dutch imperialism made the 
Dutch nation as we know it then and now - evil and cruel.\nline

When the Dutch directed their pirate ships to Indonesia, at that time their 
country was only a country of farmers and small coffee shopkeepers.\nline

Even today the country still remains as a country of farmers and merchants. 
And it will not change, because it has no basic materials for large industries, 
namely coal, iron and cotton. If the Dutch state did not have its colonies, 
it would not be able to match Belgium or Sweden.\nline

At its height it was just a country of lonely peasants and small merchants like Denmark.\nline

With the courage and will of a pirate and the greed of a small coffee shopkeeper, 
he confiscated all the Indonesian produce. There is not a single stone for indigenous 
economic housing that is left behind. How can we expect a wise government from these 
pirates, and small shopkeepers! (How can security and state management be expected of pirates and thieves - beware!).\nline

Before the arrival of the East Indies Company, the Chinese, Hindu, Arabs (eventually) 
became Javanese or at least continued to live in the country, but the Dutch came to 
Indonesia and returned to their country with sacks full of untold riches. There, 
Indonesian money was scattered and that is where they sucked their pension funds 
from Indonesian coffers. As a result, the Indonesian economy is leaking and drying up!\nline

If the Netherlands is a developed industrial state, it will sooner or later 
have to, like Britain and America, adopt a different politics.\nline

They will certainly use liberal politics against Javanese or Indo-Javanese nobles. 
Thus, political and economic progress as it is now happening in the Philippines 
and India, can also happen in Indonesia. Although the Netherlands has begun to 
industrialise Indonesia in the last 20 years, the goal remains a monopoly. Its capital remains foreign capital.\nline

The gap between the coloniser and the colonised now remains as it was in the days 
of Daendels and van den Bosch. Only the roaring voice of the revolution can fill that deep chasm.\nline

But it seems that because of this, Indonesia and other Asian countries 
will survive the imperialism defended by the Netherlands. Because of the 
sharp social conflict in Indonesia, one day a new nature will inevitably 
emerge that can free Indonesia and the whole of Asia from Western actions forever.\nline
